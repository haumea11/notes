\documentclass[11pt]{article}
\usepackage{fullpage}
\usepackage{amsmath}

\DeclareMathOperator{\arcsec}{arcsec}
\DeclareMathOperator{\arccot}{arccot}
\DeclareMathOperator{\arccsc}{arccsc}

\begin{document}
{\bf Comments:} I figured out a way to use LaTeX a little better for my maths 
notes.  Gummi (or emacs) allows for constant viewing of the pdf output so 
writing math is made a little less complex. \\
{\bf Edit:} I now have vim macros, set up to be a more than competent 
replacement for any IDE.

\section{Notes - 15.02.10}
\begin{itemize}
\item[1.]
	\begin{align*}
		\frac{d}{dx} \left[ln(8)+3ln(x)\right] &= \frac{3}{x}
	\end{align*}
\item[2.]
	\begin{align*}
		\frac{40x(5x^2-3)^3}{(5x^2-3)^4}
		&=\frac{40x}{5x^2-3}
	\end{align*}
\item[3.] {\bf Reminder Vocab:} To find the derivative implicitly, you find $\frac{dy}{dx}$ by taking the derivative with respect to x, which gives you $\left( \frac{dy}{dx} \right)$

\item[4.] {\bf Logarithmic Differentiation} is a different way to differentiate by taking the natural log of both sides.  Here is an example.
	\begin{align*}
	y &= \frac{x^2 \sqrt{5 + x}}{x^3+4}\\
	ln(y) &= ln \frac{x^2\sqrt{5 + x}}{x^3+4}\\
	&= 2ln(x) + \frac{1}{2}ln\left(5+x\right)-ln\left(x^3+4\right)\\
	\frac{1}{y}\frac{dy}{dx} &= \frac{2}{x}+\frac{1}{2(5+x)}-\frac{3x^2}{x^3+4}8
	\end{align*}
\end{itemize}

\section{Notes - 15.02.11}
\begin{itemize}
\item [61. ] This problem is a (severe) compression of the one from last night, and I skipped the notation of all the horrible algebra.
	\begin{align*}
		y^prime &= \frac{-\frac{x}{x^2+1}*x+\sqrt{x^2+1}}{x^2}
		+\frac{1+\frac{x}{\sqrt{x^2+1}}}{x+\sqrt{x^+1}}\\
		&= \frac{1+x^2}{x^2\sqrt{x^2+1}}
	\end{align*}
\item [1. ] There are some interesting things with integrals.  This set are indefinite.
	\begin{align*}
		\int \frac{x^2}{3-x^2}dx &= \int -\frac{1}{3}u^(-1)du\\
		u &= 3-x^3\\ du &= -3x^2dx\\ 
		\int -\frac{1}{3}u^(-1)du &= -\frac{1}{3}\ ln|{3-x^3}|+c\\
	\end{align*}
\item [2. ]
	\begin{align*}
		\frac{d}{dx} \Bigl[x+y-1&=ln(x^2+y^2)\\
		1 + \frac{dy}{dx}&= \frac{2x+2y\ \frac{dy}{dx}}{x^2+y^2} \\
	\end{align*}
\end{itemize}

\section{Notes - 15.02.12}
\begin{itemize}
\item [1. ] {\bf Long division} and rewriting are fun.  Here is an example of why I am lying.
	\begin{align*}
		\int \frac{x^3-3x^2+5}{x-3}\ dx &= \int \left(x^2 + \frac {5}{x-3}\right) \\ \\
		\int \frac{x^2-4}{x}\ dx &= \int \left(x-\frac{4}{x}\right)
	\end{align*}
\item [2. ] {\bf Trig functions} are also just the coolest.  338/31 you are given csc and sec generally so you don't need to worry as much, just let $ u = 2x $
\end{itemize}

\section{Notes - 15.02.18}
\begin{itemize}
	\item[1. ] {\bf Inverse Functions} are functions found when one switches x and y and solves for y.  These require the function they are based on to be 1:1, essentially passing the Horizontal Line Test.  An interesting property is that $f(f^{-1}(x)) = x$.  This also applies in reverse, $f^{-1}(f(x)) = x$.  This is the {\bf Definition of Inverse Functions}.
	
	Interestingly enough, the graph of a function and its inverse is symmetric over $y=x$.  The {\bf Reflective Property of Inverse Functions} states that all points $(a, b)$ on a function have a corresponding point $(b, a)$ on its inverse.  To figure this out, use the increasing from derivative thing.
	

	The existence of an inverse function is based on its being a 1:1 function and its being strictly monotonic on its entire domain.  {\bf Monotonic} means that it doesn't change directions.  If a function is monotonic, it is guaranteed to be 1:1 and thus will have an inverse.
		
		\item[2. ] {\bf Guidelines for finding the inverse function:  }In order to find if something has an inverse, figure out if it's 1:1 first and then figure out what its inverse is.  Finding the inverse is based on swapping x and y.
		\begin{align*}
			f(x) &= x^3-6x^2+12x\\
			f^\prime (x) &= 3x^2-12x+12\\
		\end{align*}
\end{itemize}
\section{Notes - 15.02.25}
Thus begins the reign of the great Marchi.
\begin{enumerate}
	\item Trig functions should be mechanical.  Even if the unit circle remains unmemorized, the 2 special triangles can be used to bring any special angle to bear.
	\item The sine of an angle is the output, the arcsin of the output is an angle.  Not tough.
	\item Arcsin is not a function.  In restricting the domain of it, or the range of $sin(x)$, one can make sin a 1:1 function and thus arcsin becomes a function.  II-III and I-IV are continuous.  Look at $sin(x)$ over the interval $\left[ \frac{-\pi}{2}, \frac{\pi}{2}\right]$.
	\item This brings us to the concept of ``Principal Roots.'' Sin runs between $-\frac{\pi}{2}$ and $\frac{\pi}{2}$, leaving us a 1:1 function. 
	$$ 1 - cos^2(x) = sin(x) $$ is something that helps a lot in dealing with Trig equations.  See the following:
	\begin{align*}
		2sin^2(x) &= 2 + cos(x) \\
		2\left(1 - cos^2(x)\right) &= 2 + cos(x) \\
		2 - 2cos^2(x) &= 2 + cos(x) \\
		-2cos^2(x) &= cos(x) \\ 
	\end{align*}
\item It is important to check solutions.  Some operations introduce false answers in finding roots (obviously.)
	\begin{align*}
		1-cos(x) &= \sqrt{3} sin(x) \\
		1 - 2\cos{x}+\cos^2x &= 3\sin^2x \\ 
		1 - 3 - 2\cos{x} + \cos^2x + 3\cos^2x &= 0 \\
		4\cos^2x - 2\cos x -1 &= 0 \\
		(2\cos x + 1)(\cos x -1) &= 0 \\
	\end{align*}
	However, one of the answers that this yields, $\frac{4\pi}{3}+k$, yields a false answer.  This comes from the fact we had to square at the beginning to get anywhere with what we know.  This highlights the importance of checking work.
\end{enumerate}
\section{Notes - 15.02.26}
\begin{enumerate}
	\item While the multiplicative inverse is the same as the reciprocal, the inverse and the reciprocal are different things.
	\item The graph of csc is parabolae, or seemingly so, the domain of which being all reals except $0 +- \pi k $.  Secant is the same but shifted, as expected of something cosine-related.  Cotangent is fairly similar to tangent, but differently placed.
	\item Evenness, oddness, etc: one can determine if something is even by whether $f(x) = f(-x)$.  Symmetry over the y axis also applies.  Odd functions follow $f(-x) = -f(x)$.  Sin is odd, cos is even, obviously.  Csc is odd, Sec is even.  Tan is 
	\begin{align*}
	f(t) &= |\csc t| \\
	f(-t) &= |\csc -t| \\
	\end{align*}
	\item Think about Implicit differentiation, like this:
	\begin{align*}
	9x^2+4y^2 &= 36 \\
	18x + 8y \frac{dy}{dx} &= 0 \\
	\frac{dy}{dx} &= \frac{-9x}{4y}
	\end{align*}
	\item Trick for Implicit diff: numerator, treat y as though it were a constant.  Denominator, treat x as though it were a constant.  Ask someont to explain this a little more in depth at some point.
	\begin{align*}
	4x^2y -3y &= x^3 - 1 \\
	\frac{dy}{dx} &= - \frac{8xy-3x^2}{4x^2-3}\\ 
	\end{align*}
	\item Logarithmic differentiation: it is helpful to use both logarithms and implicit differentiation to deal with all of life's problems.  For example:
	\begin{align*}
		y &= x^x \\
		\ln y &= \ln x^2 \\
		\ln y &= x\ln x\\
		\frac{1}{y} \frac{dy}{dx} &= x \frac{1}{x} + \ln x \\
	\end{align*}
	\item Logarithmic differentiation offers an alternative to the quotient rule, as well.  Instead of dealing with fractions that leave one wanting to vomit tremendously, the laws of logarithms can be applied to add and subtract things.  So instead of the fraction:
	\begin{align*}
		y &= \frac{(x-3)^4(x^2+1)}{(2x+5)^3} \\
	\end{align*}
	You can have:
	$$ \ln y = \ln (x-3)^4 + \ln (x^2+1) - \ln (2x+5)^3 $$
\end{enumerate}
\section{Notes - 15.02.27}
\begin{enumerate}
	\item You can graph $\frac{d}{dx}$ of a function on a graphing calculator.  You can also do definite integrals.  Mr. Marchi is very excited about some feature of graphing calculators he learned earlier (excited being a relative term, as I feel Marchi may only think in equations.)  Integrals take a while to graph, because they're so tricky to calculate (as graphing calculators have the processing power of an analog wristwatch.) {\bf Sto} button can be used to avoid using a rounded answer.  Press Sto and then the alpha letter you want to represent it.
	\item Storing values saves the problem of rounding, and can be helpful if a value is used again and again.
	\item For programming Riemann Sums etc. use the list menu and take a sequence.  It follows the syntax \texttt{seq(expression, variable, start, stop, scale)}, and can be stored as a list. Sto L1 lets you hit Stat then Edit to look at the lists.  List then Sum(L1) will give you the value of the Riemann Sum you input.
	\item There is a way to do that without storing it as a list, but storing it as a list makes it seem like you did the work by giving you the values.  You can do L/R/Midpt Riemann sums with the same setup and different start/end values.
	\item Trapezoid rule is trickier than the other one, because outside values are used only once.  This runs into using two lists.  Sequence is super useful.
\end{enumerate}
\section{Notes On Chapter 5}
\subsection{Section 5.5 - Bases other than e}
\begin{enumerate}
	\item Exponential Functions are functions that take the form $a^x$.  Any function taking the form of $a^x$ can also be written as $e^{\ln a^x}$.  This means $a^x = e^{a\ln x}$.  This means any exponential can be written in terms of the natural logarithmic function.
	\item The {\bf Change of Base} formula is as follows:
	$$ \log{a}{x} = \frac{\ln x}{\ln a} $$
	\item Properties of Inverse Functions are thus:
	\begin{enumerate}
		\item $y = a^x$ if and only if $x = \log{a}{y}$
		\item $a^{log{a}{x}} = x$ for all $x>0$
		\item $\log{a}{a^x} = x$ for all x
	\end{enumerate}
	\item The general formula for exponentials to a power u is:
	$$ \frac{d}{dx} a^u = u^\prime(\ln a)a^u $$
	\item The above is a very useful shortcut for the use of actual rules in problems.  Do not view it as a plain formula.  Here are some other formulae useful in similar things:
	\begin{enumerate}
		\item $ \frac{d}{dx}[a^x] = (\ln a ) a^x $
		\item $ \frac{d}{dx}[\log{a}{x}] = \frac{1}{(\ln a )x} $
		\item $ \frac{d}{dx}[\log{a}{u}] = \frac{1}{(\ln a )x} $
	\end{enumerate}
\end{enumerate}
\subsection{Defining e}
\begin{enumerate}
	\item Interesting way to define e:
	$$ \lim x \to \infty \left( 1 + \frac{1}{x}\right)^x = \lim x \to \infty \left( \frac{x+1}{x}\right)^x $$
	The book calls this "A limit involving e." It's on page 364.
\end{enumerate}
\subsection{Inverse Trig Functions}
\begin{enumerate}
	\item No trig function is 1:1, but any one of them can be limited in order 
		to talk about its inverse functions.
	\item Arcsin can be understood as ``The sin of what is equal to\dots?'' or 
		some similar concept.  In this sense, arcsin is to sin as ln is to $e^x$.  
		\begin{enumerate}
			\item $\arctan 1 = \frac{\pi}{4}$
			\item $\arccsc \frac{2}{\sqrt{3}} = \frac{\pi}{3}$
			\item $\arcsin -\frac{\sqrt{3}}{2} = -\frac{\pi}{3}$
			\item $\arctan (-1) = -\frac{\pi}{4}$
			\item $\arccsc (-2) = -\frac{\pi}{6}$
			\item $\arctan (-\sqrt{3}) = x$
			\item $\arccos \frac{1}{2} = \frac{\pi}{3}$
			\item $\arccos \frac{\sqrt {3}}{2} = \frac{\pi}{6}$
			\item $\arccos - \frac{\sqrt {3}}{2} = \frac{5\pi}{6}$
		\end{enumerate}
		You get the idea, but essentially this is the principle of it; Inverse 
		trig functions aren't that difficult.
	\item In trig, just draw triangles.  Do it.  That's all you need.  It 
		really works wonders.
\end{enumerate}
\subsection{\dots And How to Use Them.}
\begin{enumerate}
	\item $\frac{d}{dx}[y = \arcsin(x)] = \frac{d}{dx}[\sin y = x]$, hence 
		$\cos y \frac{dy}{dx} = 1$ so 
		$$\frac{dy}{dx} = \frac{1}{\cos y} = \frac{1}{\sqrt{\sin y} =
		\frac{1}{\sqrt{\arcsin x}}}$$
	\item Similarly, here's arccos.
		\begin{align}
			\frac{d}{dx}\arccos x &= \frac{d}{dx}\left[cos y = x\right] \\
			-\sin y \frac{dy}{dx} &= 1 \\
			\frac{dy}{dx} = \frac{-1}{\sqrt{1-x^2}}
		\end{align}
	\item Here are the rules for all derivatives of inverse trig functions:
		\begin{align}
			\setcounter{equation}{0}
%			\frac{d}{dx} \arctan x &= \frac{d}{dx}\left[tan y = x\right] \\
%			\csc ^2 (y) &= 1 \\
%			\csc ^2 (arcsin x)\frac{dy}{dx} &= 1
			\frac{d}{dx}\arcsin u &= \frac{u^\prime}{\sqrt{1-u^2}} \\
			\frac{d}{dx}\arccos u &= \frac{u^\prime}{\sqrt{1-u^2}} \\
			\frac{d}{dx}\arctan u &= \frac{u^\prime}{u^2+1} \\
			\frac{d}{dx}\arccot u &= \frac{-u^\prime}{u^2+1} \\
			\frac{d}{dx}\arcsec u &= \frac{u^\prime}{|u|\sqrt{u^2-1}} \\
			\frac{d}{dx}\arccsc u &= \frac{-u^\prime}{|u|\sqrt{u^2-1}}
		\end{align}
		Notice how $\frac{d}{dx \arccos u} = - \frac{d}{dx} \arcsin u$ and the 
		rest of those statements apply here.
	\item And here is some practice:
		\begin{align}
			\setcounter{equation}{0}
			y &= \arcsin x + x \sqrt{1-x^2} \\
			&= \frac{2-2x^2}{\sqrt{1-x^2}} \\
			\frac{dy}{dx} &= \frac{1}{\sqrt{1-x^2}} + \frac{-x^2}{\sqrt{1-x^2}}
			+ \sqrt{1-x^2} \\
			&= 2\sqrt{1-x^2}
		\end{align}
	\end{enumerate}
\end{document}
