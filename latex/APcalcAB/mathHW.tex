%%=====================================================================================
%%
%%       Filename:  mathHW.tex
%%
%%    Description:  The math homework, starting as of the 7th March 2015
%%
%%        Version:  1.0
%%        Created:  03/07/2015
%%       Revision:  none
%%
%%         Author:  John Markiewicz
%%   Organization:  
%%      Copyright:  
%%
%%          Notes:  
%%                
%%=====================================================================================
\documentclass[11pt]{article}
\usepackage[]{amsmath}
\usepackage{fullpage}
\DeclareMathOperator{\arcsec}{arcsec}
\DeclareMathOperator{\arccot}{arccot}
\DeclareMathOperator{\arccsc}{arccsc}
\begin{document}
\section{Section 5}
\subsection{377/5-12, 17-27 odds (Section 5.6, day 1)}
\begin{enumerate}
	\item[5. ] Find the arcsin of $\frac{1}{2}$ without using a calculator.
		$$\arcsin \frac{1}{2} = \frac{\pi}{6}$$
	\item[6. ]  $\arcsin 0 = 0$
	\item[7. ]  $\arccos \frac{1}{2} = \frac{\pi}{3}$
	\item[8. ]  $\arccos 0 = \frac{\pi}{2}$
	\item[9. ]  $\arctan \frac{\sqrt{3}}{3} = \frac{\pi}{6}$
	\item[10. ] $\arccot (-\sqrt{3}) = -\frac{\pi}{6}$
	\item[11. ] $\arccsc \left(-\sqrt{2}\right) = \arcsin \frac{1}{(-\sqrt{2})} 
		= -\frac{\pi}{4}$
	\item[12. ] $\arccos \left(-\frac{\sqrt{3}}{2}\right) = \frac{5\pi}{6}$
	\item[17. ] Evaluate without using a calculator:
		\begin{enumerate}
			\item \[\sin \left(\arctan \frac{3}{4}\right) = \frac{3}{5}\]
			\item \[\sec \left(\arcsin \frac{4}{5}\right) = \frac{5}{3}\]
		\end{enumerate}
	\item[21. ] Write in ``Algebraic'' form.
		\[\cos \left(\arcsin 2x\right) = 2x\]
	\item[23. ] Write in ``Algebraic'' form. 
		\[\sin (\arcsec x) = \frac{\sqrt{x^2 - 1}}{|x|}\]
	\item[25. ] Write in ``Algebraic'' form.
		\[\tan\left(\arcsec\frac{x}{3}\right) = \frac{\sqrt{x^2-9}}{3}\]
	\item[27. ] Write in ``Algebraic'' form.
		\[\csc\left(\arctan \frac{x}{\sqrt{2}}\right) = \frac{\sqrt{x^2+2}}{x}\]
	\end{enumerate}
\end{document}
