\documentclass[11pt]{article}
\usepackage{amsmath}
\usepackage{fullpage}

\begin{document}
\section{New Content}
\subsection{Monday, 13th April}
\begin{enumerate}
	\item Slope Fields are drawings with the slope at any given point, and can tell a 
		lot about a differential equation without solving it.
	\item A big reason slope fields are useful is that there are some differential
		equations we are unable to solve.  The handout has some examples and rules.
	\item Drawing out a slope field by hand is tedious and unpleasant, so don't
		do it unless you have to.
\end{enumerate}
\subsection{Tuesday, 14th April}
\begin{enumerate}
	\item Continuing from April Fools' Day:
		\begin{align}
			\setcounter{equation}{0}
			\frac{dy}{dx} &= 4xy \\
			\int_{}^{}\frac{1}{y}dy &= \int_{}^{}4xdx \\
			\ln y &= 2x^2+C_1 \\
			y &= Ce^{2x^2}
		\end{align}
	\item $\frac{dy}{dt}$ is the same as a rate of change, but it's also a part
		of a differential equation.
		\begin{align}
			\setcounter{equation}{0}
			\frac{dy}{dt} &= ky \\
			\int \frac{1}{y}dy &= kdt \\
			ln y &= kt+c \\
			y &= Ce^{kt}
		\end{align}
	\item {\bf Growth and Decay} gives us the equation $y = Ce^{kt}$ where 
		$C$ is the initial value, $k$ is the constant of proportionality, $e$ is
		$e$, and $t$ is time.  If $k$ is positive, it is growing, and if negative,
		it is decaying.
	\item In many problems using growth and decay, we want to first find k.
		\begin{align}
			\setcounter{equation}{0}
			4 &= Ce^{2k}\\
			4 &= 2e^{2k}\\
			2 &= e^{2k} \\
			\ln 2 &= \ln e^{2k} \\
			\frac{1}{2} \ln 2 &= k
		\end{align}
	\item Continuing on, using this fact:
		\begin{align}
			y &= 2e^{\frac{t}{2}\ln 2}\\
				&= e^{\ln 2^{\frac{t}{2}}} \\
				&= 2(2^{\frac{t}{2}} \\
				&= 2(2^{\frac{3}{2}})
		\end{align}
	\item This sort of thing can be applied as well to half-lives.
		\begin{align}
			\setcounter{equation}{0}
			y &= Ce^{24100k} \\
			\frac{y}{c} &= e^{24100k} \\
			\ln \frac{1}{2} &= 24100k \\
			k &= \frac{\ln \frac{1}{2}}{24100}
		\end{align}
		if $k$ is half-life, then $k=\frac{ln\frac{1}{2}}{h}$ and if k is a
		constant of proportionality then h is $\frac{ln\frac{1}{2}}{k}$.
\end{enumerate}

\section{AP Preparations}
\subsection{Friday, 17th April}
\begin{enumerate}
	\item 
\end{enumerate}
\end{document}
