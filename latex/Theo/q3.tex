%%====================================================================================
%%
%%       Filename:  q3.tex
%%
%%    Description:  Notes for the Third Quarter of Theology 11 with Simmers
%%
%%        Version:  1.0
%%        Created:  01/20/2015
%%       Revision:  none
%%
%%         Author:  John Markiewicz
%%   Organization:  Little Theatre, Gold Standard Choir
%%      Copyright:  
%%
%%          Notes:  Anything stated in this document has a Catholic tint to it.
%%                  This does not indicate I have any agreement with the Church.
%%====================================================================================
\documentclass[10pt]{article}
\usepackage{fullpage}
\title{``Life in Christ''}
\begin{document}
\section{Hearing God's Voice}
\begin{enumerate}
\item The secret to happiness is to draw near Jesus and learn from him.
\item Essential steps to Catholic Morality are to do as Jesus says and 
	to put into practice his example and his teaching.
\item The ``Definition'' of morality is knowledge based on human 
	experience, reason, and God's revelation that discovers what 
	people ought do to live fully human lives.  Or it's ``knowing 
	what ought to be done'' or ``The science of what humans ought 
	to do by reason of who they are.''
\item Basic to being a moral human being is appreciating the ``gift of 
	being human'' and respecting the value of human life.  Using 
	intellect and law also are to guide one's freedoms.
\item Forming, informing, and following a conscience are important.  
	Repenting and seeking forgiveness when one does something 
	wrong.  Loving God above all is important.  Loving oneself is 
	also important.  (Hey, a good message!)
\item Loving your neighbor is important.
\item ``Make yourself what you are.'' There is within the self the 
	person who one would like to be.  One has the potential and 
	the power to become the person one would like to become.
\item It is less the things one does than it is the kind of things one
	does.  People want to be popular, respected, and accepted for 
	who they are.  Apparently a clip from Wicked works here.
\item Atheism.  This is the philosophy I already know, because it's a 
	portion of mine.
\item Deism is the belief in a God but the God is no longer involved 
	(or in the case of Nietzche dead.)
\item Agnosticism, the idea God cannot be proven.
\item Dignity is the quality of being worthy of esteem or respect.  
	Every human person has worth and value.  Catholics view this 
	as being because of our creation in God's image.  Inherent 
	means it does not need to be earned or required.
\item Animals do not have ``free will,'' that is, the power rooted in 
	reason and will that enables one to perform deliberate actions 
	on one's own responsibility.  
\item Humanity has the ability to love.  Capacity to grow and responsibility.
\item Principle of Subsidiarity is the idea of not doing what one 
	below you can do as well or better.
\item Common Good is the good of all, even if it is to the detriment 
	of the few.
\item Solidarity is unity with others (duh.)
\item Kohlberg (debunked) said that people go through a series of 6 
	moral stages : Reward/punishment, Self-interest (Me first, 
        hedonism.) Others first, Legalistic, What Society Stands For,
        and the idea of principles.  Kohlberg posited a Christian form
        of this.
\item Humans are by nature social, but do enjoy time to selves.  People are a
        part of various social groups with principles of unity.
\item There is a distinct concept of ``Original Sin'' in Catholicism.  This 
  is a ``hereditary stain'' which means by some confusing means it got passed
  on?  God works in mysterious ways, apparently.
\item There are 6 steps to living a moral life: be who you are, respect
  everyone, develop and share gifts, love God, love the self, love others.
\end{enumerate}
\section{Right Reason in Action(or: Morality?)}
\begin{enumerate}
  \item Right Reason in Action is based on Prudence and Discernment.  
	  Discernment is a decision making process attending to the implications
	  and consequences od an action or choice.
  \item STOP: Search out the Facts, Think about alternatives and consequences, 
	  think about others, pray.
  \item Actions express who we are, form us into who we will be, and impact 
	  the world around us.
  \item Obviously, intention has its importance.  Teaching is that the action 
	  is only good if the action and the intention are good.
  \item Circumstance can alter the morality of something, though they cannot 
	  change the moral quality of an act; they cannot make good an evil action.  
  \item In the idea of the Catholic church, for something to be truly moral, 
	  it should have good action, good intention, and good circumstances 
	  morally.  This can't always happen, and the Church doesn't endy this.  
  \item Alternatives allow people to ``choose freely'' or something.
  \item Consider consequences, and ask if everyone in that situation should 
	  act in the same way.
  \item One must also consider the effects of one's actions on others.
  \item Consulting with others is important to decision making.  
  \item The idea is posited that the Church has some authority and power to 
	  inform Catholics in matters of morality.
  \item The Golden Rule, which in Christianity is not motivated by the actual 
	  desire to be treated by others as oneself would want to be treated, but 
	  rather by being a good person.
  \item Prayer apparently may give a different perspective whenever one takes 
	  time away to pray.
\end{enumerate}

\section{Rules, Regulations, and the Rest}
\begin{enumerate}
	\item There are many rules and regulations through life.  The regulations, 
		however, are in place for a reason.
	\item Freedom is the power to perform deliberate actions in one's own
		responsibility.
	\item Determinism is the philosophy holding that every event, action, and 
		decision results from something independent of the human will.
	\item Freedoms can be external (from outside forces) and internal (from 
		inside factors.)  Internal factors often limit the choices.
	\item Human freedom is obviously limited by a handful of things.
	\item True freedom is not license.
	\item Abuses against freedom include ignorance, inadvertance, duress, 
		inordinate attachments, fear, and habit.
	\item Responsibility is accountability linked to a person.  
	\item Emotions are morally neutral, but actions are not.  
	\item Jesus is the fundamental norm of Christian morality (obviously.)  
	\item Norms originate as beliefs.  
	\item The purpose of both general and specific norms is to direct 
		individuals and societies toward responsible behavior and right 
		action.
	\item Law is, by definition:
		\begin{itemize}
			\item Reasonable
			\item For the common good
			\item Made by competent authorities.
			\item Must be announced.
			\item Is a basic necessity for a harmonious society.
			\end{itemize}
	\item The law is in place to prevent immoral people from destroying the 
		rights of others, ensure the smooth run of society, and attempt to 
		guide the members to the fulfillment of its purpose.
	\item Natural law is the participation of humans in ``God's Eternal Law.''  
		This reveals God's intentions, or some such.
	\item There are 3 kinds of Civil Law:
		\begin{enumerate}
			\item Restrictive -- You can't do things.
			\item Directive -- You have to do things.
			\item Progressive -- Law that helps a society move forward.  Can be
				either of the previous types.
		\end{enumerate}
	\item The claim at hand is that Civil Law is only good insofar as it conforms
		to God's eternal law.
	\item The Old Law is the 10 commandments.
	\item The New Law is the Divine Law of the New Testament, guiding with four
		purposes.
		\begin{enumerate}
			\item It helps people on the path toward God.
			\item It helps discern when there are conflicting ideas of right and
				wrong.
			\item It speaks of motivation.
			\item It indicates what is sinful.
		\end{enumerate}
\end{enumerate}
\end{document}
