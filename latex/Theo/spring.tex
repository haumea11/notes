\documentclass[12pt]{article}
\usepackage{fullpage}

\begin{document}
\section{Spring Break Assignment}
\subsection{``Page 123 What Would you Do?''}
\begin{enumerate}
	\item[The Wallet] Assuming my current situation, if it's easy to get the wallet
		back to the owner I do so.  Carrying 500 dollars on oneself is foolish, but
		it's not my money to take.  Perhaps I place a small photograph of a frowning
		man in the wallet.
	\item[The Big Party] If they ask, I tell them.  I'm not going to lie to my
		parents, and I know they're going to ask anyway, so it doesn't really matter.
		But I'm not even really that interested in big parties like that anyway, so
		it doesn't really seem like something that would come up.
	\item[Sexist Remarks] I call them out on it, as I definitely have in the past.  
		I'm a pretty big feminist, and I'm not going to back down on something I 
		feel strongly about, like that.
	\item[The Test] Academic honesty is something I don't compromise on.  I would
		sooner take a zero on a test than cheat, but considering I've worked these
		problems anyway it's not a big deal.  It's sufficient to say I would turn
		down the offer.
\end{enumerate}

As a side note, I see ``conscience as a myth'' and have to question which 
figures they've found to deny the existence of a personal conscience.  Even
someone who thinks that organized religions use conscience as a tool wouldn't
try to claim that there is no such thing as conscience, just that it doesn't
come from God.

\subsection{``Page 124 in the margin in pink font under a Book Icon.''}
\begin{enumerate}
	\item Conscience as superego, in that it's from a source that's not god.  I 
		don't actually see what's so wrong with this, because God is just as much a
		source of outside psychological conditioning early in life as authority is in
		most of the United States.  In my view of morality, good should be done for
		a love of one's fellow man, not out of fear of an all-powerful deity.
	\item This is conscience as majority opinion.  This is a flawed worldview on
		so many levels, but it's surprisingly common.  Anyone with this view of
		morality should seriously reconsider what they're doing.
	\item This is the hedonistic conscience as a feeling.  There is actually
		very little wrong with smoking marijuana, but the real fact of the matter
		is it's not a moral issue.  Joe is willing to take the legal risks and 
		social stigma associated with marijuana use, and it's his choice whether
		to do that.  I thoroughly believe marijuana is a non-moral issue, even though
		the drug does not appeal to me.
	\item I'm not sure how to report on something here, but the issue of racism is
		often dismissed with a sort of conscience-by-majority-opinion thing.  It
		can be seen in modern times with the racism (and religious discrimination)
		against muslims from the middle east.  
\end{enumerate}

\subsection{page 125 in the margin in pink font under a Heart Icon.}
	I've done very, very little in the past week.  I'm going to stretch it
	to include a school week as well.  Of course, I always do my homework.  I 
	never cheat on tests, so I was honestly taking every test I took this week.  
	I've done this assignment giving my actual opinions, so that's a case of honesty
	as well.

\subsection{page 127 Stolen Sign ( there is 1 question = 5 pts each)}
	I'm a bit surprised, firstly, that there was a fatal accident at all in a 
	low-traffic rural area.  This aside, I would consider reporting them,
	mostly depending on whether I thought the punishment they would receive
	was proportional to the crime they committed.  They obviously didn't mean to
	kill someone, and shouldn't be prosecuted as if they did.

\subsection{page 125 For your Journal}
	I have often taken my conscience as a prompt to help a group make a better
	decision.  Many times, my conscience leads me to try to get people to
	avoid using homophobic, racist, or culturally insensitive terminology.  By
	telling people that what they're saying is offensive, I can do my
	part to make the world a less discriminatory place.

\subsection{page 129 For your Journal}
	I haven't had to make any truly difficult conscience decisions lately.
	Generally when confronted with difficult moral decisions, I work based on
	my principles and would say I am proud of my decision.  If it were a real
	morally difficult situation, I would not be proud because I wouldn't be
	sure if I had made the ``right'' decision.  That comes up very infrequently,
	though.

\subsection{page 131 For your Journal}
	My parents would tend to be a fairly solid choice to ask in matters 
	pertaining to morality and conscience.  They have my best interests in mind
	and tell me the truth, which is something I can trust very few people to
	do.  This aside, I would take anything anyone says with a grain of salt until
	I have formed my own opinion on an issue of morality or conscience.

\subsection{page 136 For your Journal}
	I currently have no cause for which and no person for whom I would die.  I 
	cannot, in good conscience, write a prayer to a God I don't believe in 
	asking for strength I don't want in convictions I don't have.  I think that
	if it's God's plan for someone to have to go through martyrdom, then God
	is a spiteful God to his own people.

	Moreover, I think it is damaging to inspire this kind of blind willingness
	to die for a cause in adolescent persons (likely, might I add, given to them
	as {\it children},) and expect them to have it on faith.  This kind of system,
	one that trains children to be soldiers willing to die for an idea, would not 
	be acceptable	(let alone commonplace) in any other field than religion.  I 
	think that asking teenagers to compose a prayer, asking for strength in facing
	death for an idea, is reprehensible and I am stunned that someone would think
	to call this morally just.

\end{document}
