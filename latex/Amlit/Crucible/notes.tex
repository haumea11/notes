\documentclass[10pt]{article}

\usepackage{fullpage}

\begin{document}
\section{Background Information}
\subsection{American Drama}
\begin{itemize}
	\item Eugene O'neill was the first ``important'' figure in American Drama.  
	\item Introduced realistic drama with the removal of the viewer by the 
		fourth wall.
	\item Arthur Miller and Tennessee Williams represent Realism and Realism 
		combined 	with an attempt at something more imaginative.
	\item Miller wrote Death of a Salesman, All My Sons, and The Crucible.
\end{itemize}

\subsection{The Crucible}
\begin{itemize}
	\item Produced in 1953, about puritans and the salem witch trials.  
		This was in parallel to McCarthy's Red Hunt.  God, I hate McCarthy.
	\item There was a belief in Salem in witches, seen as the Devil's helpers,
		and	confession led to freedom from the devil's bonds.  Confusion in 
		identifying	witches came from shape-shifting and possession.
\end{itemize}

\section{Act One}
\begin{itemize}
	\item Reverend Samuel Parris is a main character, play opens in his room.
		He kneels by his daughter's bed, she is lying inert.
	\item Act opens with a digression, and a discussion of puritans.  One
		point he makes about these people tend to mind others' business, and
		that this feeds the witch trials.
	\item The puritans believe that they light the world.  This has helped
		and hurt America in the time since.
	\item Miller says that the Salem Witch Trial developed from a paradox
		in whose grip we still live.  The issue is one of the theocracy, but
		the balance turning toward individual freedom.
	\item We return to a mid-forties Parris, who is a bad man.
	\item Tituba is a slave, brought from Barbados by Parris from when he
		was a businessman there.  Believed to be able to conjure spirits.
	\item Abigail is strikingly beautiful, an orphan living with Parris
		who is her uncle.  She is very dramatic.  She tends to seem like
		she's lying here, because of the way she's acting.
	\item Someone was running naked in the forest.  Abigail may have drank 
		blood.  They danced.
	\item Betty is lying unresponsive in bed, tries to fly when she wakes
		up.  She actually passed out, probably.
	\item Mercy is a merciless.  She's the Putnam's servant, is sly.
	\item Mary is subservient, naive, and lonely.  She's a bit of a foil
		to Mercy.
	\item Mrs. Putnam tends to be full of canards.
	\item Abigail says she was discharged from Goody Proctor's service
		because she hates Abigail and has it out for her.  People want 
		slaves, not servants, so she hasn't gotten a job lately.
	\item It is assumed that the actual reason is that Abigail has
		had an affair with Goody Proctor's husband John Proctor.
	\item Ann Putnam has had many babies die.  She seems to be the gossip
		of the town, and simply makes matters worse.
	\item Thomas Putnam is a well-to-do, hard-handed land-owner and is
		50.  A commentary follows; he is a man with many grievances, at 
		least one of which appears to be justified.  His brother-in-law
		got turned down for a position he should have gotten.  Putnam is
		the eldest son of the richest man in the village.  His vindictive 
		nature was demonstrated long before the witchcraft happened at all.
	\item Putnam feels his own name and the honor of his family have been 
		besmirched by the village.  He feels he was unfairly treated in his
		father's will.  He accuses a lot of people.
	\item Putnam dislikes Parris, and Parris is not entirely aware of this.
	\item Putnam gives Parris advice to strike out against the devil,
		and this may just be a ploy to mess Parris up.
	\item Betty awakes, and tries to fly out the window.  Abigail drank
		blood, and Betty makes note of this.
	\item The girls danced, Tituba tried to conjure the spirits of 
		some dead sisters or something.  Abigail threatens the hell out of
		the other girls in case they do tell.
	\item ``In Proctor's presence, a fool felt his foolishness instantly.''
	\item Proctor is respected, and feared, but he feels like a fraud.
	\item Proctor has a specific attitude toward Abigail, and she's 
		pretty into him.  He still thinks about her, but states he will
		never touch her again.
	\item Rebecca Nurse and Giles Corey enter.
	\item Nurse is really calm, chill.  Her husband is well-respected,
		and she is as well.  She's been through a lot, and is viewed as
		wise.  She's very gentle, by nature.
	\item Putnams and Nurses have a land dispute.
	\item Putnam accuses Rebecca's spirit of tempting her to iniquity,
		and this is more true than Putnam knows.
	\item Nurse has 11 children, 26 grandchildren, and have seen all
		of them go through their foolishness.  She has a lot of experience
		with children and says you have to let them be kids.
	\item Parris has sent for Reverend Hale, who apparently has some
		expertise in the field of witches and devils.  Rebecca says he
		should just be sent back as soon as he gets here, saying that the
		Reverend will just foster further disagreement.  Mrs. Putnam seems
		to resent and envy Rebecca Nurse who has had so many living children
		and grandchildren.
	\item There are arguments about land, deeds, Parris's preaching,
		firewood, factions in the church.  This is a town full of conflict
		and contention.  Reverend Hale enters.
	\item The ideals of the world haven't changed since the days of strong
		religion even though there is a superficial change.
	\item Hale says not to look to superstition, and yet this is a witch
		trial play.
	\item Putnams tend to be witchcraft believers, Abigail, the other girls,
		etc.  Parris doesn't want witchcraft, and Proctor, Nurse, and maybe
		a couple others just don't believe it.
	\item Giles is wacky, comical, gets blamed for a lot, isn't very good
		at praying.  Was a crank, and a nuisance, but is deeply innocent and
		brave.
	\item Hale questions the girls, especially Tituba and Abigail.  
	\item Girls try to push the blame off themselves, for the most part
	\item More small talk is made, and a sense of separation rises.  
		Elizabeth doesn't want conflict but she must cause it anyway.
	\item 
\end{itemize}
\end{document}
