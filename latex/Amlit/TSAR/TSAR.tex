%%=====================================================================================
%%
%%       Filename:  TSAR.tex
%%
%%    Description:  
%%
%%        Version:  1.0
%%        Created:  03/05/2015
%%       Revision:  none
%%
%%         Author:  YOUR NAME (), 
%%   Organization:  
%%      Copyright:  
%%
%%          Notes:  
%%                
%%=====================================================================================
\documentclass[11pt]{article}
\usepackage{fullpage}
\usepackage[]{amsmath}
\begin{document}
\section{Background and First 3 Chapters}
\subsection{Hemingway}
\begin{itemize}
	\item Hemingway came up with the Hemingway code, summarized by the following ideas:
		\begin{itemize}
			\item Live Passionately
			\item Do not be controlled by others
			\item Be honest, controlled, and disciplined
			\item Take risks, be persistent
			\item Do not pity yourself, and accept death on your own terms.
		\end{itemize}
	\item Fitzgerald and Hemingway write at the same time, but there are many 
		important differences.  Like Fitzgerald, Hemingway got involved with 
		other writers.
	\item Hemingway uses a simple sentence structure and reveals fairly little.
	\item Hemingway also uses irony, sarcasm, and understatement.  So 
		Hemingway should remind me of myself in some twisted way.
	\item Hemingway was born in Oak Park, Illinois, to a doctor.  He loved the 
		woods, boxed, and played football.  Like many other writers of the 
		time (I'm looking at Fitzgerald here) he was involved in WWI.  He was 
		rejected by the US Army, as a result of eyesight.
	\item Hemingway, rather than fighting, served as an ambulance driver in 
		Italy.  Wrote \underline{A Farewell to Arms} about it.  
	\item In Paris, he met other writers including Ezra Pound, Gertrude Stein, 
		and F. Scott Fitzgerald.
	\item These American authors living in Paris were called ``expatriates,'' 
		and Stein named the generation of writers ``The Lost Generation.'' 
		Stein had an influence on Hemingway, advising his style and telling 
		him to concentrate more.
	\item In 1923, Hemingway saw his first bullfight in Pamplona, Spain.  He 
		used his experiences with bullfighting in writing \underline{The Sun
		Also Rises.}
	\item Published TSAR in 1926, and it was a critical success.
	\item Also wrote \underline{Farewell to Arms}, \underline{For Whom the 
		Bell Tolls}, and \underline{The Old Man and the Sea}.
	\item Received the Nobel Prize for literature.
	\item Uses the quote from Stein, and a quote from Ecclesiastes
\end{itemize}
\subsection{Backdrop of the Book}
\begin{itemize}
	\item Jake Barnes, our narrator, is a WWI vet in the newspaper business.  
		He was injured in the war.
	\item Robert Cohn is a Jewish man from a wealthy family, and is apparently 
		a mediocre writer (assuming we can trust our narrator.)
	\item Lady Brett Ashley is Jake's love aged 34.  She plans on getting a 
		divorce from Lord Ashley.
	\item Bill Gorton is Jake's friend met in Spain and a writer.
	\item We find Cohn has little self-confidence and is controlled to a fair 
		extent by the women in his life.
	\item Cohn has been discriminated against for being Jewish, and while he 
		hates boxing he hides behind it and uses it to feel safer.
	\item Cohn was from one of the richest and oldest Jewish families in New 
		York.  Robert was nice, friendly, but shy.  It made him bitter.
	\item Braddocks is Cohn's literary friend, and Barnes is his literary 
		friend.  Cohn wrote a novel, but it was a critical failure.
\end{itemize}
\subsection{Chapter 1 through Chapter 3}
\begin{itemize}
	\item Cohn had been reading \underline{The Purple Land} and is taking it 
		too literally.  Jake thinks Cohn needs to look more realistically at 
		life.  Cohn needs to live life rather than having all these Romantic 
		delusions.  Jake believes that nobody ever lives life to the fullest 
		except for bullfighters.  Robert is not so interested in bullfighters, 
		and Jake says he should read a book about it.
	\item Robert wants to see a different life than he has, Jake is satisfied 
		with what he's doing.  Jake has tried going to other countries, and 
		says ``You can't get away from yourself by moving.  I've tried all 
		that.''  This is the idea you can't run away from death or yourself.
	\item We are introduced to Lady Brett Ashley, and she was ``Damned good 
		looking.''  She wears her hair in the style of the 20s, short and 
		brushed back ``Like a boy's.''
	\item There is evidence she and Jake are more than just casual friends.  
		She dodges Robert's attempts at flirtation.  At her entrance Jake 
		immediately points out her presence, and is insulted that he brought 
		Georgette.  She calls him darling.
	\item Ends the chapter saying she is miserable.  She acts carefree, but
		is unhappy with the circumstances.
\end{itemize}
\section{Chapters 4-6}
\subsection{Chapter 4}
\begin{itemize}
	\item Brett is unreliable, and portrayed as not knowing what she wants and 
		speaks differently from how she acts.  
	\item Brett is assertive and independent, but ``afraid of so many 
		things.''  Jake claims he can see through her exterior.
	\item ``Don't we pay for all the things we do?'' -Jake
	\item ``When I think of the hell I've put chaps through, I'm paying for it 
		all.''  -Brett
	\item In response to this, Jake says his war injury is supposed to be 
		funny, and that he never thinks of it.
	\item Jake has grown tired of the subject, though ``Certain injuries are a 
		source of merriment'' but are a pain for those with them.
\end{itemize}
\subsection{Chapter 5}
\begin{itemize}
	\item Zizi and Mippipopolous.  Mippi knows a lot about people, and owns a 
		chain of sweet shops.  Brett says he's ``one of us.'' This suggests he 
		is aristocratic, from America and living in Europe, frivolous, and all 
		the like things.
	\item Jake takes this walk through the city streets, walking past cafes.  
		Comes to a statue of Marshall May, a French military hero known for
		his bravery.
	\item Jake has a very important relationship with bullfighting, he takes 
		in bullfighting papers or something.  He mentions how the Catholic 
		Church says he shouldn't think about his permanent injury.
	\item Jake lays awake thinking about Brett, and cries.  He wants to be 
		with her, but feels there is no chance.  They go out again, and kiss, 
		and he goes back upstairs, gets into bed, and says ``it is awfully 
		easy to be hard-boiled in the daytime, but at night is another thing.''
	\item He throws out the idea that Brett isn't even worth crying about.
\end{itemize}
\subsection{Chapter 6}
\begin{itemize}
	\item Jake has a conversation with a couple other newsmen, at a press 
		conference.  Jake does not consider these men friends, just 
		acquaintences.  They don't know each other very well.  The other 
		newsmen are married and have kids.
	\item Cohn has an incapacity to enjoy Paris.  Jake thinks he got this from 
		the writer H.L. Mencken.  He implies this is where Robert gets many of 
		his ideas. 
	\item Jake encounters Harvey Stone, and they see Robert Cohn coming and 
		Harvey is kind of a dick.  Says Cohn is an unintelligent moron.  Then 
		later says he is a case of arrested development.
	\item Cohn takes this pretty well, but threatens Stone a bit.  Jake notes 
		he feels he has not shown Robert clearly.  Up until Cohn fell in love 
		with Brett, he had never made a remark that would detatch him from 
		others.  Externally, he had been formed at Princeton but internally he 
		had been formed by women in his life.
	\item Francis is upset that Robert refuses to marry her.  Jake dislikes 
		how Robert allows Francis to walk all over him.
\end{itemize}
\end{document}
