%%=====================================================================================
%%
%%       Filename:  gatsbyoutline.tex
%%
%%    Description:  The outline for the in-class paper
%%
%%        Version:  1.0
%%        Created:  02/25/2015
%%       Revision:  none
%%
%%         Author:  John Markiewicz
%%   Organization:  
%%      Copyright:  
%%
%%          Notes:  
%%                
%%=====================================================================================
\documentclass[11pt]{article}
\usepackage{fullpage}
\usepackage{outlines}
\usepackage{enumitem}
\setenumerate[1]{label=\Roman*.}
\setenumerate[2]{label=\Alph*.}
\setenumerate[3]{label=\roman*.}
\setenumerate[4]{label=\alph*.}
\begin{document}
\begin{raggedleft}
John Markiewicz \\
February 22nd, 2015 \\
Period 1, Bucher \\
In-Class Essay outline \\
\end{raggedleft}
\noindent
\\ THESIS STATEMENT: In \underline{The Great Gatsby}, F. Scott Fitzgerald uses 
the automobile as a symbol of the danger and hypocrisy of the wealthy in the 
1920s through the damage they do, the attitudes the characters have toward 
them, and how the characters interact with them.
\begin{outline}[enumerate]
		\1 Introduction
		\1 The damage automobiles do throughout the novel is a symbol 
	of the damage done by the lavish lifestyles and unsustainable 
practices of the rich in the 1920s.
		\2 The owl-eyed man, in a drunken state, drives a car into a 
	ditch and very weakly denies that it was his fault.
		\2 While Tom is happy to drive through using Gatsby's car, he 
	is quick to dissociate from it as soon as it is found to have been the 
car Myrtle was killed with.
		\2 Daisy ends up killing someone using Gatsby's car, and 
	Gatsby is murdered as a result, making an incredibly symbolic point 
about why Gatsby wanted to be wealthy and what it got him.
	\1 Characters demonstrate the attitudes toward wealth through their 
attitudes toward automobiles and driving.
		\2 Tom recklessly speeds through town, and when caught by a 
	cop demonstrates who he is and gets by with no trouble.
		\2 Nick, while he does little to challenge it, is concerned by 
	the way the people around him deal with driving.
		\2 Jordan drives so recklessly, she would be worried to run 
	into someone as careless as herself, yet she makes no attempt to fix this.
	\1 The way characters interact with automobiles is symbolic of their 
relationship with wealth.
		\2 Tom tries to use cars to keep Daisy away from Gatsby, but 
	like his wealth this does not work and leads to Gatsby driving off 
alone with Daisy in Tom's car.
		\2 The Wilsons, living in comparative poverty in the valley of 
	ashes, work (even when George is very ill) to keep cars running and 
the rich happy, especially highlighted by Tom's demanding attitude toward George.
)	\2 Gatsby owns a car, and has others drive in it often, but he doesn't 
actually drive his own car himself during the novel, demonstrating the way in 
which Gatsby approaches his own wealth not for his own enjoyment but for the 
approval and enjoyment of others.
	\1 Conclusion
\end{outline}
\end{document}
