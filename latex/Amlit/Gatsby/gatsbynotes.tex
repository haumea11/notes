\documentclass[12pt]{article}
\begin{document}
\section*{Background on Gatsby}
\begin{itemize}
\item Fitzgerald views Gatsby as a condemnation of the rich society he 
  himself is intricately involved with.  The rich spend money on 
  elaborate parties and expensive acquisitions.
\item Profits are made legally and illegally, and postwar era has a 
  whirlwind pace to it.
\item The ability to mass-produce goods led to ``stabilization'' and 
  prosperity.  The term ``The Jazz Age'' was coined by Fitzgerald.
\item Jazz Age began soon after WWI and went to the crash of 1929.  
  Economy boomed, women became more independent socially and
  economically.
\end{itemize}

\section{The First Chapter}
\begin{itemize}
\item Nick as a narrator is somewhat reliable but is not completely 
  so.  He lies, but only in omission.  He is biased but honest
  about it.  Tries not to judge Gatsby.  Says Gatsby is 
  everything for which he has an unaffected scorn, but that 
  Gatsby turned out alright in the end.
              
\item Description of Gatsby is vague, mysterious in this chapter.
\item We learn of Nick that his house is a small eyesore in the 
  proximity of West Egg millionaires.  The summer began when he 
  went to the Tom Buchanans.  
\item Nick goes on to explain the relationship between Tom and Daisy 
  Buchanan.  Cousins with Daisy, went to college with Tom.  
\item Daisy and Jordan are there on the couch, the ``only stationary 
  object in the room.''
\item Describes Daisy's voice in a bit of detail, people don't forget 
  it.  She is depressed but putting on a very good act (?)
\item Daisy says ``That's the best thing a girl can be in this world, 
  a beautiful little fool.''  In this, she sees that women are 
  forced to be subordinate and wants her daughter to be able to 
  deal with that.  She doesn't want her daughter to have to be 
  as cynical and miserable as she.
\item Daisy has turbulent emotions, thinks everything is terrible, 
  says she has experienced everything and was sophisticated.  
  But she says this insincerely.  
\item Nick sees across the lawn Jay Gatsby who reaches off for the 
  (incredibly symbolic) green light and trembles, so Nick 
  doesn't call to him and then Gatsby just vanishes.
\end{itemize}

\section{The Second Chapter}
\begin{itemize}
\item The valley of ashes, gray depressing place full of ash.  T.J. 
  Eckleburg's eyes on a billboard are blue, gigantic, and behind 
  enormous {\bf yellow} glasses.  This is full of symbolism in 
  addition to being a setting.
\item Tom wants Nick to meet his mistress, which is kind of weird.  
  Everyone knows that Tom has a mistress.  Tom gets a bit drunk, 
  and insists on Nick following along.  The only building in 
  sight is the small yellow building at the edge of the 
  wasteland.  This has Wilson's car shop.
\item People are kind of asses to Wilson.  Tom assesses George as ``so 
  dumb he doesn't know he's alive.''
\item Myrtle, faintly stout but sensuous without beauty in her face 
  but a perceptible vitality about her.
\item Myrtle is somewhat of a foil to Daisy.
\item Tom invites Nick (much to Nick's dismay) to his apartment, and 
  they get drunk off their asses.
\item Myrtle changes dresses, and she transforms (becomes fake.)
\item The remainder of the chapter follows the party at the apartment.
\item The party in chapter 1 was a lot more formal but personal.  The 
  main difference between the two parties is in the fact that 
  people are far more honest at the party in chapter 2.
\item The parties also share drama from Tom's combination of women.
\item The first party has awkwardness, and lower emotions, where at 
  the apartment feelings seem to be running higher.
\end{itemize}

\section{The Third Chapter}
\begin{itemize}
\item This chapter focuses on the lavish parties thrown by Gatsby, and the one of them that Nick is invited to.
\item Chapter opens and describes the amount of citrus Gatsby goes through every weekend.  Several Crates, enormous amounts of orange juice.  This is because of the scale of Gatsby's parties.
\item Nick is invited to the party, which is something unusual in its own right, as people usually just show up or are brought by someone else.
\item Owl-eyed man, having been drunk for a week, is in the library because he thinks it may sober him somewhat.  He makes note of the fact that the books in the library are real, but that they haven't been read as the pages are still connected at the ends (as books used to be when they were new)
\end{itemize}
\end{document} 
