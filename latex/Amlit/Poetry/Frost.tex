\documentclass[10pt]{article}

\usepackage[margin=1in]{geometry}

\begin{document}
\section{Background on Frost}
\begin{itemize}
	\item Wrote many poems.  Attempted to capture the sound of common speech in 
		his poetry, but still manages to sound eloquent.
	\item Poems should ``Begin in delight and end in Wisdom.''
	\item Frost won 4 Pulitzer Prizes.
\end{itemize}
\section{Questions and Mending Wall}
\begin{itemize}
	\item Walls are not loved by their nature (?)
	\item There is no concern for the fact a human built a wall, and walls are
		unnatural barriers.
	\item There are things that don't love a wall, and destroy a wall.
	\item Speaker and neighbor meet yearly to repair the wall between them, and 
		the neighbor thinks good fences make good neighbors.
	\item Walls are things that separate, isolate, keeps someone or something 
		else out.
	\item ``We keep the wall between us as we go.''
	\item The wall is a barrier, it is important to know what one is walling in
		or walling out.
	\item The neighbor got the saying about walls from his father, and moves in 
		darkness.
	\item Neighbor gets compared to an old-stone savage.  The speaker is
		suggesting that the neighbor doesn't understand.
\end{itemize}

\section{Out, Out---}
\begin{itemize}
	\item This poem's title takes from Macbeth's speech in Macbeth.  The title is
		significant in that the poem speaks on how life is brief.  
	\item Setting contrasts ending, as a beautiful scene.
	\item Snarling and rattling of the buzzsaw is threatening and is a case of
		foreshadowing.
	\item Boy gets distracted or something, and his hand moves toward the saw and
		has not a hand.
	\item The boy is a big boy doing a man's work, and is worked hard.
	\item Boy dies, and since nobody remaining is the one dead, they return to 
\end{itemize}
\end{document}
