\documentclass[10pt]{article}
\usepackage[margin=1in]{geometry}
\usepackage{setspace}

\begin{document}
\begin{raggedleft}
	John Markiewicz\\
	23 March 2015\\ 
	Am. Lit. Acc with Bucher\\
	College Essay\\
\end{raggedleft}
\begin{center}
	Why I Wear a Suit
\end{center}
\doublespacing

	People usually have idiosyncracies, details about themselves
	that distinguish them from others.  A lot of these don't do
	anything to others' opinions, or actually influence them 
	negatively.  Some aspects of individuals, however, make
	a person different from the accepted norm in a good way.  I
	started wearing formal clothing as a challenge to the idea
	that it is uncool to do so, and it has been a benefit to 
	me in several ways.

	Generally speaking, we as a culture associate the suit
	with businessmen (and businesswomen.)  This gives it a whole
	set of connotations; people have an image spring to mind
	every time the word ``businessman'' comes up in conversation.
	On one hand, business is associated with bureaucracy, 
	boredom, dissatisfaction, and of course work.  On the 
	positive side, though, business is associated with success, 
	money, drive, and power in its more positive connotations.
	My goal in wearing a suit was to take advantage of the
	positive aspects of the businesslike image without 
	evoking the negative aspects, in an attempt to build an
	image.

	Originally, I had challenged the idea that suits are boring
	as a response to a boring situation.  I had just made the
	move to high school, and was getting used to the ins and
	outs of my dress code.  I had become pretty bored with the
	routine of the days.  One Friday, since there was no rule
	against it, I wore a suit to school.  The look caught on, 
	and so every Friday thereafter I wore a suit to keep the
	tradition going.

	People responded surprisingly well to my initial move
	toward formal clothes.  It was partially as a result of
	the suits that I got my first nickname, ``Crazy John.''
	Regardless of one's perspective on how positive a thing
	this nickname is, it was given with endearment.  As a 
	result of how people responded, and how comfortable I
	have become with suits, I would absolutely challenge the
	idea that suits are boring and uncomfortable again.

	Suits are one aspect of me that differs from an accepted
	cultural norm.  Really, though, it's best to question any
	cultural norm if it isn't backed by some reason.  In
	questioning and challenging the things in our life that we
	accept as fact, we can improve our lives and have a better
	time of things.
\end{document}
