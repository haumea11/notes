\documentclass[11pt]{article}
\usepackage[T1]{fontenc}
\usepackage[rm]{roboto} %% Option 'black' gives heavier bold face

\begin{document}
\title{Notes on \underline{Raisin}}
\author{John Markiewicz}
\date{13th May 2015}
\maketitle
\section*{Background on Lorraine Hansberry}
\begin{itemize}
	\item Grew up on the South Side of Chicago.  Had an early interest in writing,
		and her most notable work \underline{A Raisin in the Sun} opened in 1959.  
		It was a big event, and sparked a vigorous movement in Black Theatre.
	\item Eugene O'Neill said ``A play should focus on the most intense and basic
		human relationships,'' which \underline{Raisin} does.
	\item The play focuses on a black family, the Youngers.
	\item The play is one about a family, but is also one dealing with larger 
		sociological issues.
	\item Her husband, after her death, collected much of her work into an 
		informal biography---``To Be Young, Gifted, and Black''.
\end{itemize}
\section{Introduction and Act I}
\subsection*{Introduction}
\begin{itemize}
	\item The introduction to the play (the version given to us, anyway,) is one
		on the versions and impacts of the play.  It criticises the earlier TV
		version of the play for removing important content.
	\item The original cuts made to the play were, in the perspective of the 
		author, necessary and took little away from the meaning of the play.  In
		order for the play to show on Broadway in the manner it did, some revision
		was necessary.
	\item The play makes comments about race, and about racism, and about the 
		segregation of neighborhoods and life at the time.
	\item One of the biggest selling points of \underline{Raisin}, the author of
		the introduction states, is that the Younger family is like any other 
		American family.
	\item The capacity of people to ``Deceive themselves where race is concerned''
		is mentioned, with regards to people in general and Northerners.
\end{itemize}
\subsection{Act I, Scene I---Friday Morning}
\begin{itemize}
	\item Family shenanigans occur.
	\item Walter and Beneatha are at odds, and Walter feels worried.  He both 
		worries about her, and worries about his own potential.
	\item There are many arguments.  Walter implies Ruth is never on his side, and
		says Mama would listen to her, then trying to get her to talk about the
		liquor store to Mama.
	\item Mama's entry is built up.  
\end{itemize}

\section{Act II}
\begin{itemize}
	\item Asagai and George are interesting people in Beneatha's life, and 
		represent the differences in what she wants and what people want for her.
	\item George's parents are successful and wealthy.
	\item Walter is kind of a dick to George, and would not be able to converse
		with his parents.
	\item Beneatha is saying OCOMOGOSAI and Ruth is like Pearl Bailey.
	\item When Walter leaps up on the table he is said by Beneatha to be a
		descendant of Shaka.
	\item The statement is made on p. 86 between George and Walter, George calls
		Walter Prometheus, which is an allusion.  Walter does not do as he is told.
	\item Ruth complains of black people.  Walter asks who even cares about her.
	\item Ruth apologizes for the baby, and asks what else she can give him.
	\item Ruth is sick of the apartment, and her goal is to provide a better life
		for her family.
	\item Walter's reaction has been a thing.
	\item George tells Beneatha she looks nice, and that guys aren't going to go 
		for the atmosphere.  He tells her to drop the garbo routine (an allusion to
		an actress, Gretta Garbo.  She oft played dramatic, hard to get characters.)
	\item George does not want to discuss with Beneatha the nature of quiet
		desperation.  This is an allusion to Thoreau.
	\item The scene with Mrs. Johnson was cut from the original play.  She is a
		woman who decided to be excited about everything.
	\item Johnson overstays her welcome.
	\item There are contrasts between Johnson and the Youngers---Johnson thinks 
		being a chauffeur is a fine job, while the Youngers do not.  The Youngers 
		hold themselves with a higher sense of pride than Mrs. Johnson.
	\item Johnson says the one racial slur what white people ought not say.
	\item Walter has been skipping work, and borrowed Willy's car and went all 
		over.
	\item Mama says there is nothing worth holding on to if it will destroy her
		boy, and she gives Walter the money.
	\item Walter has all his dreams set up and talks of how everything is going
		to be just swell, saying he will hand his son the world.
	\item Walter's dream is unrealistic.
	\item Scene 3 opens a week later, everyone is packing up and Beneatha enters
		with a guitar case.
	\item Linder shows up, being linder and saying one thing when he means another:
		\begin{itemize}
			\item Says he just wants to talk and communicate, when he wants them
				out of the neighborhood.
			\item Says race doesn't even enter into it, when that's the only thing 
				into it at all.
			\item Says black families are generally happier when they're in their own
				communities when that's not the youngers' goal.
		\end{itemize}
	\item Linder is incredibly uncomfortable with the whole scenario.
	\item Linder states it would be a financial gain to the Youngers, and once
		Walter realizes he kicks Linder out.
	\item Mama talks of the symbolic plant, and the gardening tools and hat are
		given to Mama.
	\item Mama hits Walter, as he basically gave away his father's work.
\end{itemize}

\section{Act III}
\begin{itemize}
	\item At curtain, there is a solemn light of gloom in the living room---grey
		light, not unlike that in act I.
	\item Life as a circle speech is given, referring to the lack of progress in
		the world.  Specific to Beneatha is that this refers to African-Americans
		and women.  She feels freedom and equality are false promises and there is
		little to no progress.
	\item Asagai shouts at Beneatha that ``I live the answer'' and gives a speech.
		He feels life is more of a line than a circle and that people are making
		progress even if it's slow.  As they continue to talk, he makes the point to
		her that she shouldn't give in to repressive circumstances.
	\item The idea is that it is important to take action to improve one's own 
		situation and improve it.
	\item Beneatha is viewed as giving up too easily, over the loss of the money.
	\item The whole situation, especially Asagai's comments, speaks to the themes
		of responsibility, courage in the face of difficulty.
	\item Beneatha asks if Walter was dreaming of yachts on lake Michigan.
	\item Beneatha says about the movers that there's no need in them coming out
		and going back.  That costs money.
	\item Mama recalls how she was chided as a child for being an overachiever and
		having big dreams.  This apparently runs in the family.
	\item They've hit the ``Real, honest-to-God rock bottom.'' 
	\item Walter says he'll feel like a \underline{man} when he can afford the
		sort of things linder can buy.
	\item Beneatha says Walter is not her brother.
	\item Mama says to Linder that her son says they're going to move.
	\item Mama says Walter has come into his own like a rainbow after the rain.
\end{itemize}
\end{document}
