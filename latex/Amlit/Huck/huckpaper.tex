\documentclass[12pt]{article}
\usepackage[margin=1in]{geometry}
\usepackage[doublespacing]{setspace}
\setlength{\parindent}{45pt} % Default is 15pt.

%-Setup Over-%

\begin{document}
\begin{singlespace}
\begin{flushright}
John Markiewicz\\
November 13th, 2014\\
Period 1\\
\end{flushright}
\begin{center}
Twain's Society and Huck's Freedom
\end{center}
\end{singlespace}

	\underline{Huck Finn} focuses heavily on elements of society and freedom.  Mark Twain examines these themes through the perspective of the protagonist, Huck, and uses the story to convey his message.  Often, Twain does so in a humorous or satirical manner.  \underline{Huck Finn} satirizes society in its portrayal of Huck's treatment in school, Huck's difficulty when with criminal types, and the fact Huck is only truly free on the raft.

	Huck's time in school is used by Twain as a way to examine society's limiting influence on the freedom of the individual.  Huck is introduced to the schooling process and feels like he ``couldn't do nothing but sweat and sweat, and feel all cramped up'' (4).  Huck understands that the two women he is in the care of have good intentions, but is still left feeling stifled and limited by their rules and attempts to influence him.  Huck says ``Miss Watson she kept pecking at me, and it got tiresome and lonesome'' (5).  Overall, Huck missed his old way of life, and the freedom of such a lax environment.  Even living in a house with a bed is considered stressful to him, as he says in the line ``Living in a house, and sleeping in a bed, pulled on me pretty tight, mostly, but... I used to slide out and sleep in the woods, sometimes, and so that was rest to me'' (19).  All of these quotes are somewhat humorous, as it is difficult to imagine someone so accustomed to living with nothing that they would find sleeping in a bed stressful, but they all point to the idea that Huck feels uncomfortable while in the care of the women who are trying to civilize him, and more importantly he feels as though he is not free.

	Huck finds life with his father (and later, his experiences with the other criminals the King and the Duke,) to be just as bad and limiting as his experiences with the women of the novel.  Huck is constantly being controlled by his father, as demonstrated by the fact that ``[Pap] kept me with him all the time, and I never got the chance to run off'' (28).  Even though Huck far prefers the way Pap lives, he still is not able to live life how he wants and finds it difficult to enjoy, in no small part because Pap is abusive; Huck states at one point ``[Pap] got to going away so much, too, and locking me in''(29).  The problem Huck has with his father is the same he has with the Widow, but they manifest in very different ways; both view Huck as lesser and try to control him.  The other influences that try to control and manipulate Huck are the King and the Duke, both of whome Huck is suspicious of and wants to escape so he can return to freedom.  When he feels like he has finally escaped them, Huck realizes ``it did seem so good to be free again... here they come! ... it was the king and the duke'' (239).  Pap, the King, and the Duke are all examples of satire, as it applies to the criminal element of the world.

	Ultimately Huck can find freedom in the one place he is viewed as an equal, on the river with Jim.  The travel down the river with nobody telling him what to do or how to live is Huck's ideal.  Huck enjoys his time on the raft with Jim, shown when he says `` `Jim, this is nice,' I says. `I wouldn't want to be nowhere else but here.' '' (58).  Unlike most of the adults in the book, Jim doesn't really try to do anything to Huck, or treat him as someone to be manipulated or changed.  In this sense, Huck is free not just because he has agency, but also because he is not subject to the demands of others.

	\underline{Huck Finn} explores the limiting influences of society and other individuals on freedom, and what freedom truly means, and does so through satire and characterization.  Huck has no real goal except to be free, and through the book he strives to that goal.  The novel uses this to show that one will lose freedom as soon as one holds himself/herself to the expectations of others.  In this way, and in some others, \underline{Huck Finn} teaches a message about life through deep characters.
\end{document}
