% Created 2015-09-04 Fri 13:50
\documentclass[11pt]{article}
\usepackage[utf8]{inputenc}
\usepackage[T1]{fontenc}
\usepackage{fixltx2e}
\usepackage{graphicx}
\usepackage{longtable}
\usepackage{float}
\usepackage{wrapfig}
\usepackage{rotating}
\usepackage[normalem]{ulem}
\usepackage{amsmath}
\usepackage{textcomp}
\usepackage{marvosym}
\usepackage{wasysym}
\usepackage{amssymb}
\usepackage{hyperref}
\tolerance=1000
\date{\today}
\title{penny-lab}
\hypersetup{
  pdfkeywords={},
  pdfsubject={},
  pdfcreator={Emacs 24.5.1 (Org mode 8.2.10)}}
\begin{document}

\maketitle
\tableofcontents

\section{Hypothesis}
\label{sec-1}
The digital scale is more accurate and more precise than the
analogue scale.

\section{Data}
\label{sec-2}
\begin{center}
\begin{tabular}{lrlll}
\hline
Penny Letter & Year & Mass (digital) & Mass (analog) & D$_{\text{a}}$\\
\hline
A & 1995 &  & 2 mg & \\
\hline
B &  &  &  & \\
\hline
C &  &  &  & \\
\hline
D &  &  &  & \\
\hline
E &  &  &  & \\
\hline
F &  &  &  & \\
\hline
G &  &  &  & \\
\hline
H &  &  &  & \\
\hline
\end{tabular}
\end{center}
% Emacs 24.5.1 (Org mode 8.2.10)
\end{document}