\documentclass[11pt]{article}

\usepackage[T1]{fontenc}
\usepackage[rm]{roboto}

\renewcommand{\arraystretch}{1.5}

\begin{document}
\section{Unit 1}
\subsection{Change and the Scientific Method}
\begin{itemize}
\item The change in the Crab Nebula was formed by an exploding star.  This
    was found throught the \underline{Scientific Method}.  
\item To understand change, one must understand the types of change.
  \begin{itemize}
    \item Faster, or slower.
    \item Visible, or invisible.
    \item Destructive, or peaceful.
    \item Lasting, or fleeting.
    \item Monumental, or small.
    \item Physical, or chemical.
    \item Biological, or inert.
  \end{itemize}
\item The scientific method is a method for studying change.
\item There are steps, obviously.
  \begin{enumerate}
    \item Observe and question
    \item Research and gather information
    \item Form a hypothesis
    \item Experiment, to design a test for the hypothesis
    \item Data and Analysis, in which experiment is run, and data is gathered
            and organized.
    \item Conclusion.
  \end{enumerate}
\item A theory is an hypothesis proven valid under all conditions and
    scrutiny.
\item A law is a theory proven even more valid.  It is ``infallible'' which
    is a super loaded term and I disapprove.
\end{itemize}
\subsection{Scientific Notation}
\begin{itemize}
	\item Pretty simple stuff.  A ``Standardized way of expressing values in
		science, consisting of significant figures and a power of ten as well as a
		unit.''
	\item Addition and subtraction go with decimal, Multiplication and division
		run with lowest sig figs.
\end{itemize}
\subsection{Data}
	{\bf Data} is Quantified attributes or characteristics of an item in an event
	of change---length, mass, time, or force.

	\begin{center}
	\begin{tabular}{||| c || c | c | c | c |||}
		\hline \hline
		MKS		& Meter				& Kilogram	        & Second		& Newton \\
		\hline \hline
		CGS 	        & Centimeter	                & Gram			& Second		& Dyne	 \\
		\hline \hline
		B.E.	        & Foot				& Slug			& Second		& Pound	 \\
		\hline \hline
	\end{tabular}
	\end{center}

	\begin{itemize}
		\item Precision is data aligning with data, accuracy is data aligning with 
			the truth.
		\item Measurement is a comparison to a standard.
	\end{itemize}
\subsection{Dimensional Analysis}
\begin{tabular}{||| c || c |||}
	\hline \hline
	Quantity                        & Dimension \\
	\hline \hline
	Distance 			& L \\
	\hline \hline
	Area 			        & L$^2$ \\
	\hline \hline
	Volume 				& L$^3$ \\
	\hline \hline
\end{tabular}
\[ x = vt + x_{0} \]
\subsection{Orders of Magnitude}
    \begin{itemize}
     \item An order of magnitude is a rough estimate designed to be accurate to within
       about 10.
     \item Orders of magnitude are useful, but really a rough estimate.  Greater than
       5 means round up.
    \end{itemize}
\subsection{Errors and Uncertainties}
    {\bf Systematic errors} cause a random set of measurements to be spread about a
    value rather than being spread about a value rather than being spread about the
    accepted value.

    It is a system or an instrument value.
    \begin{itemize}
      \item Badly made instruments.
      \item Poor calibration.
      \item Et cetera--
    \end{itemize}
\subsection{Random Errors}
    Random Errors are due to variations in performance of the instrument and the
    operator.

    Examples:
    \begin{itemize}
      \item Vibrations, convection
      \item Misreading
      \item Variations in the surface measured
      \item Less sensitivity when a more sensitive measurement is available
      \item Human parallax
    \end{itemize}
    To get rid of random error, taking multiple readings and averaging the range of
    results can help.

\subsection{Accuracy}
    Accuracy is the indication of how close a measurement is to the accepted value
    indicated by the relaive or percentage error in the measurement.  It has low
    systematic error.

\subsection{Precision}
    Precision is an indication of the agreement among a number of measurements and
    is indicated by the {\bf Absolute Error}.  Precision implies low random error.
\end{document}
