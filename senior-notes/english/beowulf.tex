\documentclass[11pt]{article}

\usepackage[T1]{fontenc}
\usepackage[rm]{roboto}

\begin{document}
\title{Beowulf Notes}
\author{John Markiewicz}
\date{28 August 2015}
\maketitle
\section{About Beowulf}
\subsection{What is an Epic?}
	An \underline{epic} is a long narrative poem focusing on a hero.  Generally
	there is a flaw in the hero.  Traditional, or \underline{primary} epics, are
	the ones originally told and kept orally.  Literary epics, by contrast, are
	written originally.

	Interestingly enough, there is no true epic from the United States.  This,
	however, is the viewpoint of purists.  Many say we don't need an epic, since
	the cultures moving in had their own epics.

	Since U.S. culture is different from European culture, however, there are a
	couple of works debatably classifiable as the American Epic:
	\begin{itemize}
		\item Huck Finn - very American literature, and while Huck is not
			superhuman, he is definitely American and reflects the time.
		\item The Leather Stocking Tales - About the colonists and their exploits.
			Quintessentially American as a result of its themes of colonization.
		\item Leaves of Grass - is a poem, with the nebulous "I" as the hero.
	\end{itemize}
\subsection{Notable Epics}
	\begin{itemize}
		\item The Iliad
		\item The Odessey
		\item Paradise Lost
		\item The Epic of Gilgamesh
		\item Beowulf
	\end{itemize}
\subsection{Background on Beowulf}
	The hero Beowulf represents Anglo-Saxon ideals, as many epics focus on a
	superhuman hero focusing on the cultural ideals of the time.  There was a
	Beowulf, and some of the events portrayed did truly happen.

	This doesn't mean the epic is true.

	The events portrayed happened in late 400s to early 500s, notably the burial
	of kings.  This is the earliest point \underline{Beowulf} could have been
	written.  The latest it could have been was 835 AD, and the first known copy
	of it written has been dated to 1000 AD.

	The original writing was different from the written version, because
	Christian monks 
\end{document}
