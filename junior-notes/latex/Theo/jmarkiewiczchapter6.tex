\documentclass[12pt]{article}
\usepackage{fullpage}

\begin{document}
\title{Chapter 6 Homework, Theology 11}
\author{John Markiewicz}
\date{24 April 2015}
\maketitle
\section{For your journal questions on p 151 and 160}
\subsection{p. 151}
	I really, truly don't get this one.

	So my boss gives my coworker and I equal amounts of geese.  He tells
	us to use the geese wisely, as geese are highly valuable.  In my life,
	I invest my geese properly in the goose market, while my coworker
	decides to squander his geese on rare bread.  My boss promotes the both
	of us, and I ask why he promoted the other worker, to which he responds,
	``I mean, he said he was sorry and invested a single gausling in a 
	goose mutual fund, so I figured he deserved it as much as you.''

	I don't know, what was I supposed to write?

\subsection{p. 160}
	This passage seems to be fishing for a response where I state why I 
	felt wrong and either ``I felt wrong because of God'' and it was the
	right reason or ``I felt wrong because of a non-God reason'' and it 
	was the wrong reason.

	Most of the points in my childhood where I felt bad were a combination
	of fear of authority and the sense that if someone else did what I had
	done to me, I would be unhappy.  These are not Catholic motives, and I 
	would agree that fear of authority is not a truly good reason to feel
	bad about one's actions, but the golden rule is in Christian theology
	as much as it is in my philosophy.

\section{p. 156}
	Obviously, how grave one's actions are is dependent on how the people
	affected feel about the action.  Another influencing factor is the
	intent of the action.  A related thing is whether the person performing
	the action knows what they are doing is hurting another, or whether it
	is wrong.

	One example of the lessening of the seriousness of an action would be 
	the difference between murder and manslaughter.  Neither of these acts
	is something to be taken lightly, but the lack of intent involved in 
	manslaughter is something that lessens its wrongness to a degree.
	Another example would be killing someone in justified self-defense, which
	is excusable, while killing someone without good reason is absolutely
	inexcusable.

\section{p. 157}
	Side note before I do the section: The ``Baby As Trash'' margin section
	would not have happened in a society where abortion is safe, legal, and
	destigmatized.  Just saying.

\begin{enumerate}
	\item M
	\item M
	\item M
	\item V
	\item M
	\item V
	\item N
	\item V
	\item M
	\item N
	\item N
	\item N
	\item M
	\item M
	\item N, even catholicism doesn't view the feeling itself as sinful.
	\item M
	\item V
	\item M
	\item V
	\item N, unless you are in a committed relationship in which case it
		is M.
\end{enumerate}

Although to be fair I don't view things on a mortal/venial scale.

\section{p. 159}
\begin{itemize}
	\item Tell them the truth.  Apologize.  Make up for it in any way you
		can.
	\item Re-do the work yourself.  If you are truly sorry, you will not
		cheat again.
	\item Not a problem.
	\item Community Service with the said minority group.
	\item ``Using'' them?  Apologize, do what you can to make things better
		for them.
	\item Share resources with the poor.
\end{itemize}
\end{document}
