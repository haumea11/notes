\documentclass[11pt]{article}
\usepackage[T1]{fontenc}
\usepackage[rm]{roboto} %% Option 'black' gives heavier bold face
\begin{document}
\title{Not Southern Gothic}
\author{John Markiewicz}
\date{11 May}
\maketitle

\section{Katherine Anne Porter}
\subsection{Background}
\begin{itemize}
	\item Known mainly for short stories, lived in Indian Creek TX
	\item Wrote \underline{Ship of Fools}
	\item Among her focuses are the burdens of past love.
	\item Her work is skillfully crafted, tightly structured, and written in a 
		clear, elegant style.
	\item Wrote stream-of-consciousness.
	\item Tells stories with vivid sense of scene.
	\item Stream of Consciousness is a style of writing that portrays the inner 
		workings of a character's mind.
\end{itemize}
\subsection{The Jilting of Granny Weatherall}
\begin{itemize}
	\item Jilting is oft used in the context of someone not showing up to a 
		marriage.
	\item Granny's mind is where the majority of the story unfolds, and Porter
		uses the limited Point of View.  
	\item The reader is plunged right into the stream of consciousness.
	\item Granny is dying, can't hear, is not very lucid, but tells him to leave 
		anyway.
	\item The opening page establishes Granny's character, age, and physical
		condition.
	\item Her bones feel loose and floaty, and she sees Dr. Harry as like a baloon
		floating about.
	\item She asks where he was forty years ago when she could have used a doctor,
		and says she is well as she is.
	\item Waving goodbye is too much trouble for Granny.
	\item Granny is suspicious of others.
	\item Cornelia whispers around doors, keeping things secret in a public way. 
		She is the tactful, kind, dutiful daughter.
	\item Granny always ran a clean, tidy, neat house.  She was a motivated person
		and after her husband passed away she was a widow raising the kids.
	\item The details of housekeeping provide metaphors for how Granny approaches
		life.
		\begin{itemize}
			\item Making a neat bed, for example, is a metaphor for leaving nothing
				that should be done undone.
			\item Dust-free clock and carefully arranged row of jars also serve some
				symbolic purpose or something.
		\end{itemize}
	\item As the story continues, one idea leads to another and Granny ends up 
		thinking of death.
	\item She thinks of people looking through the letters in the attic.  All the
		letters there make her uneasy because she does not want her children to see
		how foolish she was.
	\item Death feels clammy and unfamiliar in her mind.
	\item She has spent so much time preparing for death, there is no reason for
		her to take it up again in thought.
	\item She takes comfort in a memory of when she was sixty and didn't die in
		spite of being ill.  She feels she can now focus on more important matters,
		like\ldots
	\item Her father, who lived to one hundred and two years of age. He had drunk 
		a noggin of strong hot toddy on his last birthday and he claimed it was his
		secret to living.
	\item She decides to plague Cornelia.  The thing that annoys Granny about
		Cornelia is that Cornelia thinks Granny is ``Deaf, Dumb, and Blind.''
	\item Granny thinks to go back to her own house so nobody will remind her 
		constantly that she is old.
	\item This brings her to the thought of children, and how her husband would
		be younger than the children now.
	\item She feels proud of the house she ran.
	\item A fog rises over the valley ``like an army of ghosts.''
	\item It would then be time to light the lamps.  She prays, in a sense like
		her children when they wait for her to light the lamp in that she is waiting
		for death in much the same way.
	\item To have the lamps lit is comforting, and both Granny and her kids are 
		praying in a sense for comfort.
	\item This brings Granny to memory of when she was jilted.  For sixty years
		she prayed against remembering him and against going to hell.
	\item A voice in her head tells her to stand up to the jilting, and she
		decides she wants to see George again to see the life she's made.
	\item She thinks of her children: Cornelia, Jimmy, Dead Hapsy.
	\item Hapsy died at a young age, and the whole thing gets confusing.
	\item Hapsy was her favorite.
\end{itemize}
\section{T. S. Eliot}
\subsection{Background}
\begin{itemize}
	\item No 20th century poet has had greater critical esteem than T. S. Eliot
		and he has had an incredible influence on other writers.
	\item Born Thomas Stearns Eliot, went to Harvard and wrote poems while there.
	\item Wrote of J. Alfred Prufrock a couple of times.
	\item When WWI broke out, moved to England and met Ezra Pound
	\item Wrote \underline{The Wasteland}.
	\item Won a Nobel Prize for literature.
\end{itemize}
\subsection{Prufrock}
\begin{itemize}
	\item Separation of the individual from the society.
	\item The poem is considered to be a dramatic monologue, and is written stream
		of consciousness style.
	\item A synecdoche is a figure of speech in which a part of a thing is used to
		stand for or suggest the whole.
	\item An epigraph is a quotation or motto at the beginning of a chapter, book, 
		short story, or poem that makes a point about the work.
	\item The epigraph is from Dante's \underline{Inferno}, making a point of the
		fact prufrock is speaking of his own personal hell.
	\item The poem reflects ideas about Eliot's own time:
	\begin{itemize}
		\item People are spiritually bankrupt
		\item Contemporary life is unromantic and unheroic
	\end{itemize}
	\item The poem opens with a proposition of a journey through city streets.
		The destination is unclear.
	\item Speaker's state of mind is reflected in the way he perceives the evening 
		and the part of the city being passed through.
	\item The first simile is the idea of the evening spread against the sky like
		a patient etherized upon a table.  This is talking of the evening as an 
		environment unresponsive to events and emotions.
	\item Prufrock is reluctant to say what his overarching question even is.
	\item The women mentioned talking of Michelangelo.
	\item The October night is described as soft (synesthesia.)
	\item The yellow fog is described as though it were a cat.
	\item Passage of time is remarked upon, and there may be some significance to
		the approaching of winter or something.
	\item There will be time, he says.
	\item Prufrock seems self-conscious of indecision, revision, and time.
	\item It is noted that people are insincere with those who they meet.
	\item This all leads back to the room, which seems closer now.  He continues 
		to speak of the time.
	\item The stanza has Prufrock growing increasingly insecure, wanting to 
		interact with others but something is keeping him back.
	\item The question ``Do I dare?'' implies he wants to do something bold at the
		party.
	\item Prufrock is concerned about and self-conscious about his appearance, but
		he does not have an attractive appearance.
	\item He has measured out his life with coffee-spoons, suggesting a dull life
		with possible repetitive motifs.  
	\item Prufrock recalls being scrutinized by women at other parties.  The image
		painted is one of a live insect that has been classified, labeled, and 
		mounted for display.
	\item His days are compared to smoked cigarettes.
	\item ``I've known the eyes'' is a synecdoche for knowing women.
	\item Another image of himself is made to be a lobster or a crab.
	\item All these comparisons suggest Prufrock lacks self-esteem.
	\item An allusion is made to John the Baptist, with the head on the platter
		and all that.  
	\item The eternal footman (likely death) has been seen to hold his coat and 
		snicker.
	\item Another allusion is made, this time to Lazarus.
	\item Nerves may be Prufrock's inner thoughts for all to see.  He compares
		these to patterns illuminated by a magic lantern on a screen.
	\item He fears people will ignore or dismiss his sensibilities.
	\item He is ``Not Prince Hamlet.'' He views himself as a supporting character
		rather than a starring one in life.  He thinks of himself as almost being
		the fool at times.
	\item He grows old, and asks if he dares to eat a peach.
	\item Prufrock has seen the mermaids, but feels they are indifferent to him.
		Despite this, he is held by the romantic notion they represent.
	\item Prufrock's longing, in the end, is unfulfilled.
	\item The sense is given that Prufrock will never actually ask his overarching
		question---that which was mentioned at the beginning of the poem---as he has
		not by the end of the poem.
\end{itemize}
\end{document}
