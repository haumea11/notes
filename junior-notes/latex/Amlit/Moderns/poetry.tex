\documentclass[10pt]{article}
\usepackage{fullpage}

\begin{document}
\section{Masters and Robinson}
\subsection{Robinson}
\begin{itemize}
	\item Grew up in gardiner, Maine.  Descendant of Anne 
		Bradstreet, wrote narrative poems.  Had a wit to him,
		and was accomplished.
	\item Poems are set in Tillberry Town.  His poetry
		``Anticipates'' Masters's book.
\end{itemize}
\subsection{Edgar Lee Masters}
\begin{itemize}
	\item Born in Kansas, wrote free verse.
	\item Published \underline{Spoon River Anthology}, a
		set of 250 epitaphs spoken by the inhabitants of 
		a cemetery in the fictional town of Spoon River.
	\item Richard Cory - a put-together man who had
		a lot of money and then killed himself.  Had some
		serious depression, despite being refined.
		The poem uses Cory to make a point about money
		and happiness and all that.  Cory wears a mask.
	\item Miniver Cheevy - believes he was born in the
		wrong time period.  As Carraway put it, ``You can't
		relive the past.''  He has had a bad time of life,
		and sees the present era as uncultured.  He longs
		for the times of Thebes, and Camelot, and the
		Trojan War.  Missed the ``medieval grace of iron
		clothing.''
	\item Butch Weldy - a man who settled down and then
		got an explosion.  Didn't even get compensation.
	\item Thomas Rhodes is a recurring character in 
		the poems.  Ran the church, the store, and the
		bank.
	\item Fiddler Jones - Starts with 40 acres, ends
		with 40 acres, no more rich.  Enjoyed life and
		has no regrets.
	\item Coonie Potter (hardworking farmer) and 
		Redhead Sammy (local musician)
	\item Petit, the poet was based on a real friend
		of Masters's who wrote several undistinguished
		books of poetry.
	\item He's blind to everything around him his whole
		life, and his poetry isn't all that great.
	\item Richard Bone carved epitaphs, and as he grew
		more knowing of the people around him realized the
		things he carved were false.  Compares himself
		to an historian who is ignorant or corrupt.
	\item Mrs. George Reese - Husband worked at the 
		bank, and is unjustly in prison thanks to Thomas
		Rhodes's unscrupulous son.  Acts her part anyway.
	\item Lucinda Matlock - values family highly, says
		it takes life to love life.  Works hard, but 
		loves her work.  
\end{itemize}

\section{The Harlem Rennaisance}
\subsection{Background}
\begin{itemize}
	\item Harlem Rennaisance had black writers writing
		in conventional and unconventional forms.
	\item Important portion of Black history, and of
		literary history.
\end{itemize}

\subsection{Langston Hughes}
\begin{itemize}
	\item Was a poet in school, and continued on to 
		write dialect poetry in the style of Paul
		Lawrence Dunbar, and free verse in the style 
		of Carl Sandberg.
	\item Wrote a large amount of things.  Fascinated
		by the sights and sounds of Harlem, living in 
		NYC and supporting himself as best he could.
	\item Published :
		\begin{itemize}
			\item The Weary Blues (1926)
			\item Fine Clothes to the Jew (1927)
			\item Shakespeare in Harlem (1942)
			\item Fields of Wonder (1947)
			\item One-Way Ticket (1949)
			\item Montage of a Dream Deferred (1951)
			\item Ask Your Mama: 12 Moods for Jazz  (1961)
			\item The Panther and the Lash: Poems of Our 
				Times (1967)
		\end{itemize}
	\item First African-American to earn a living
		solely from writing.
	\item I, Too 
		\begin{itemize}
			\item is a protest on Jim Crow laws
			\item Starts with a nod to Walt Whitman.
				This is fairly damn clever.
			\item[2. ] The speaker uses the image of
				the kitchen and the table to describe
				the state of racial inequality in 
				America, demonstrating that the Black
				population is downtrodden and viewed as
				lesser in American culture, but soon they
				will be viewed as equals and be ashamed
				for how they behaved.
			\item[3. ] The speaker expects to move 
				from the Kitchen to the Table by becoming
				strong enough that nobody would tell him
				to move to the Kitchen when company comes,
				which is a parallel to the etc.
		\end{itemize}
	\item Harlem
		\begin{itemize}
			\item This is the Dream Deferred poem, and
				is influenced by Bebop Jazz of the 1940s.
			\item Asks what happens to a dream deferred,
				uses similies to explore the idea.
			\item Asks if it explodes in the last line.
				This implies there will be some event if
				a dream is deferred too long.
		\end{itemize}
	\item The Weary Blues
		\begin{itemize}
			\item Draws from the blues, which was a good
				movement and was primarily African-American
				before we stole it and ran with it for a 
				few decades.
			\item Attempts to capture the musicality of
				the blues with a sad raggy tune and the 
				rhythm in words.
		\end{itemize}
	\item Manhattan
		\begin{itemize}
			\item Love of the city, no missing of nature.
			\item Uses figurative language in line 10,
				saying her crowns etc talking about the 
				city with personification.
		\end{itemize}
	\item
		\begin{itemize}
			\item Idea that sorrow is across everyone,
				linked, more prominent than joy, and must
				be shared by all.
			\item Uses metaphors and similies.
			\item There is no isolation from sorrow.
		\end{itemize}
	\item If We Must Die
		\begin{itemize}
			\item Speaks of dying with dignity
			\item Structured as a sonnet, first 8
				lines speak of the current situation
				and nobility in death, last 6 lines talk
				of the future and hope.
			\item Talks of hope and of not backing
				down.
			\item Claude McKay wrote this in response to 
				violence against blacks in 1919.  It was
				written to a black audience from a view 
				of unity.
			\item Monsters is a metaphor for the white
				racists.
		\end{itemize}
	\item Dunbar - We wear the mask
		\begin{itemize}
			\item Speaks of the mask we wear, why we
				wear it, and how it feels.
			\item The mask refers to a false claim of
				contentment, and it hides the pain that
				racism causes blacks.
		\end{itemize}
	\item I Know Why the Caged Bird Sings
		\begin{itemize}
			\item Dunbar is the speaker, 3 stanzas with
				7 lines apiece, ending with similar lines
				about the caged bird.
			\item Extended metaphor for the anti-black
				discrimination and prejudice.
		\end{itemize}
\end{itemize}
\end{document}
