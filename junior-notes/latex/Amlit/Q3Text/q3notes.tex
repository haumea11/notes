\documentclass[10pt]{article}
\usepackage{fullpage}

\begin{document}
\section{Walt Whitman}
\begin{itemize}
	\item 1855 Whitman wrote \underline{Leaves of Grass}
	\item Sent a copy to Emerson, who thought extremely highly of it.
	\item Emerson said "I greet you at the beginning of a great career."
	\item Optimism, vitality, freedom of expression, love of nature show 
		through in his poems.  Much of this can be associated with 
		Transcendentalism and Romanticism.
	\item Introduced free verse to American poetry.
\end{itemize}
\section{Terms}
\begin{description}
	\item[Free Verse:] Poetry that does not conform to a consistent rhyme scheme or meter.
	\item[Catalog:] A list of things, people, or events.
	\item[Vignette:] A brief literary description or dramatic sketch.
	\item[Cadence:] The rise and fall of the words as they are spoken.
	\item[Dramatic Monologue:] A poem in which the character speaks to one or more listeners.
	\item[Anaphora:] Repitition of a word or words at the beginning of successive lines, clauses, or sentences.
\end{description}
\section{I Can Hear America Singing}
\begin{itemize}
	\item Each person in the poem is singing, and expresses their own 
		individuality, ``what belongs to him or her,'' and nothing 
		else.  In this poem and this catalog of people doing work.  As 
		they work during the day they sing their song.  People party 
		after they work.  
	\item Tone in this poem is very optimistic, has a positive outlook on 
		America.  
\end{itemize}
\section{Song of Myself}
\begin{itemize}
	\item The opening poem to Whitman's book, and Whitman is the first 
		person speaker, sounding like an observant speaker.  Speaks to the reader.  
	\item Announces the poem's topic in the first line, and announces that 
		everything belonging to him belongs to the reader implying 
		that we are all connected and one.  Very much an opening poem.
	\item Answers 3 questions: Who the speaker is, What his intentions 
		are, How he will compose his song of self-celebration.
	\item First Question: The speaker is Whitman.  He is 37, in ``Perfect 
		health,'' is born ``here'' meaning in America, of parents also 
		born in America.  Their parents were also Americans.
	\item Ends by speaking of his own writing, says he is writing poetry 
		and has intentions to get his point across in his poetry, and 
		will hold creeds and schools in abeyance, which means he will 
		put aside these specific ways of doing things.  He is already 
		doing this by instead of writing in some form of verse he uses 
		free verse.  Says he will use nature and energy as inspiration 
		in the last line.
	\item \bf{Catalogue of Grass, 6}
		begins with a child asking what the grass is, then goes on to 
		describe it with several metaphors, saying it may be the 
		hendkercheif of the Lord, a child, the uncut hair of graves, 
		or a hieroglyphic.  
	\item Uses parallelism and repitition.  In the extended metaphor of 
		the hair of graves, he basically includes all people.
	\item Says that death is not a permanent thing, that people should not 
		fear death, and that life goes on forever somewhere in the 
		universe.  In this way it is similar to \it{Thanatopsis}
	\item \bf{Catalogue of Death}
		Return in death to nature, return to Earth in death.
	\item Uses imagery to discuss the diffusal of the self into nature, 
		speaks of the fact his message and his spirit will live on after he dies.
\end{itemize}
\section{American Realism}
\begin{itemize}
	\item By mid-1800s the north was based on manufacture and commerce, 
		while the South remained dependent on slaves and agriculture.  
		States had to start facing the issue of rights of individual 
		states within a Democratic Union.  Abolitionists became more 
		and more vocal with regards to slavery.  
	\item The growth of cities and doubling of America's population 
		through immigration from abroad both occured during this 
		period.  \underline{The Liberator} and \underline{Uncle Tom's 
		Cabin} are written and enjoy a popularity.
	\item In 1860, honest Abraham Lincoln becomes the 16th president of 
		these United States.  In 1861, the Civil War begins, which 
		lasts until the surrender at Appomattox in 1865.
	\item Following the Civil War, America had some pretty significant 
		issues.  The South had to go through Reconstruction, which 
		took a while and did less than it wanted to.  After the war 
		the trans-continental railroad, the telephone, the lightbulb, 
		and the horseless carriage all see invention.  Well, 
		technically the railroad itself was invented long before, but 
		the one across the continent was implemented then.
	\item Whitman's free verse is a serious contribution of the time.
\end{itemize}

\begin{description}
	\item[Realism:] A literary movement producing literature which depicts 
		life as it is actually lived.
	\item[Regionalism:] A literary movement that focused on the special 
		atmosphere or local color of a particular area.
	\item[Local Color:] The use of specific details describing the 
		dialect, dress, customs, and scenery associated with a 
		particular region or section of the country.
	\item[Naturalism:] A literary movement portraying people caught within 
		forces of nature and society beyond their understanding or control.  
\end{description}

\section{Abraham Lincoln}
\begin{itemize}
	\item 1809-1865, Four Score and Seven Years Ago\ldots
	\item Born in a log cabin in Kentucky in 1809, grew up in a poor 
		family in Indiana.  Had almost no formal schooling, instead 
		educated himself via reading.  In 1830, moved to Illinois and 
		became interested in politics.
	\item Believed that as president one of his chief duties was 
		preservation of the Union, dreamt that the nation be reunited.  
		The war was won, and was assassinated by Booth in the Ford 
		Theatre 5 days later.  
	\item Gettysburg Address delivered in November, at the dedication of 
		the National Soldiers' Cemetery at the Gettysburg battlefield.  
		Lincoln has a style, and uses repitition.
	\item Speech was a rally around the idea of winning the war.  
		Structure in 3 parts: What has led to the scenario, what 
		cannot be done, and what must be done.  Past/Present/Future.  
		At the start of the speech, talks of the birth of the nation.  
		Speaks about the people who have died, and of a possible 
		rebirth for the nation.  A new birth of freedom and a devotion 
		to the principles for which the soldiers died.
	\item Uses formal diction, rhythm, and is very eloquent.  Uses 
		parallelism and repitition as a means of inspiration.  Speech 
		is heavily emphasized by a lot of its aspects.  Definitely 
		speaks to the sense of unity between Lincoln and the people gathered.
	\item The use of the word ``dedicate'' through the speech
	\end{itemize}
\section{Frederick Douglass}
\begin{itemize}
	\item ``You will see how a man was made a slave, you will see how a 
		slave was made a man.''
	\item Born into slavery in the Auld house in Baltimore, where he 
		learned to read.  On his second attempt to escape in 1838, he 
		succeeded and was free.  Published autobiographies, and was a 
		powerful abolitionist speaker.  Founded The North Star, an 
		abolitionist newspaper.  Was influential in the war and was 
		even an advisor to Lincoln.  Got sent to a plantation and it 
		was decided that his spirit needed to be broken.  Covey, 
		renowned for being a slave breaker, is kind of a dick.  
		Douglass describes the first 6 months of his stay with Covey.  
		Douglass admits that Covey succeeded in breaking him.  The 
		first 6 months were apparently the worst.
	\item Douglass goes back to Master Thomas's, and has a conversation 
		with him.  This shows perseverence.  Thomas listens to him, 
\end{itemize}

\section{An Occurence at Owl Creek}
\begin{itemize}
	\item Ambrose Bierce, author of this work, fought in the civil war, 
		moved to San Fransisco and became a successful journalist.  
		Became known for writing satirical stories.  Is now primarily 
		known for short stories.  Also wrote a satirical novel, poems, 
		and essays, apparently.  Deeply affected by the glorification 
		of war and its wastefulness.  At 71 years old, went to report 
		on the Mexican Revolution, and never was seen again.
	\item The third part of the story is not real, definitely.  Story 
		opens with description of soldiers, spectators, lieutenant at 
		the right end of the line.  ``Death is a dignitary who when he 
		comes announced is to be received with formal manifestations 
		of respect, even by those most familiar with him.''
	\item Narrator then draws focus to the man being hanged.  A planter, 
		about 35, owns slaves, has a family, and supports secession.  
		While we do not know what he did, we know he was no ``vulgar assassin.''
	\item Narration shifts to limited omniscience.  Looks down and sees 
		madly swirling stream, but driftwood in the stream is moving 
		pretty slowly.  This is strange.
	\item Watch is symbolic.  There is an indication here that things 
		aren't what they seem to be.  Ticking of his watch bothers him 
		much, does not enjoy anticipating death.  Sound is called the 
		tolling of a deathknell, and as his fear increases he 
		perceives them more strongly.  Story goes to a flashback.
	\item Man is ardently devoted to the Southern cause, but cannot fight 
		with the army.  Wants to do something to be a part of it.  Is 
		a civilian who is at heart a soldier.  Encounters a man who 
		seems to be a confederate soldier.  He is being hanged because 
		of the driftwood thing.
	\item Part II goes back to objective.
	\item Part III opens, he loses consciousness and is as one already 
		dead.  Feels himself suffocating.  We come to know the rope is 
		broken and he is fallen into the stream.  He thinks to himself 
		that being hanged and drowned is not so bad, but being shot 
		would be unfair.  
	\item Many details of the third part are unrealistic, for example the 
		fact Farquahr is able to sense things far more clearly than he 
		should be hearing.  This especially applies to sounds, which 
		he hears with an amazing degree of clarity.  He hears fish 
		sliding though the water.  Too much detail is visible, he has 
		superhuman strength, things that can't happen keep happening.  
		Finds his neck horribly swollen and in pain.
\end{itemize}

\section{Stephen Crane}
\begin{itemize}
		\item War is Kind and all that.
		\item Naturalism plays a role, portrayal of Fred Collins is 
			naturalistic, he is an ordinary character who speaks 
			in the dialect of the time, and shows that people's 
			lives are affected by natural and social forces.
		\item Collins's physical needs take precidence over other 
			concerns, proving that some force other than his will 
			is at work in the story.  
		\item Story implies that heroic behavior, as it is a mystery, 
			is not easy to define.  Crane employs a lot of irony 
			in the story, often involving the chaos and brutality 
			of war, which become ironic when contrasted with 
			certain situations in the story.  Final irony, at the 
			end of the story, is that after all of his effort the 
			water is wasted.
		\item This brings it back to naturalism, and brings focus to 
			the futility of human effort to control some events.
		\item Uncertainty lies behind much of human behavior.  Heroism 
			is elusive and often beyond our understanding, but 
			ordinary people have the ability to reach beyond 
			themselves and act heroically. 
\end{itemize}

\section{Kate Chopin}
\begin{itemize}
	\item Began writing in St. Louis, MO.  Wrote short stories, concerning 
		the lives of the French Creoles in Louisiana.  Wrote The 
		Awakening, which was met by hostility in the Victorian 
		American time period, and contained themes of the repression 
		women were facing.
	\item Story takes place at the turn of the century, women were faced 
		with great amounts of limitation.
	\item The Story of an Hour covers issues of feeling free, living as 
		one's own person and not being controlled by others, and other 
		themes of being glad your husband is dead.
	\item Main character of the story has a heart problem, did not hear 
		the death of her husband with disbelief but instead wept and 
		went to her room alone. Open window symbolic of a new 
		beginning, of a new freedom.  Sees different things happening 
		outside, and feels new.
	\item Whispers the words ``free, free, free.''
	\item Sometimes loved her husband, but felt repressed.  She died of 
		heart disease, of ``joy that kills.''
	\item She was finally free of her husband, but as fate would have it 
		he was actually not dead and she dies.  Technically frees her anyway.
	\item Themes of relativity of time, that things are not always as they 
		seem, and the irony of fate are all developed in the story.  
		Is an example of realism depicted through the ordinary 
		individual's reaction to a normal event.
\end{itemize}
\end{document}
