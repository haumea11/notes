\documentclass[12pt]{article}
% \usepackage[T1]{fontenc}
% \usepackage[rm]{roboto} 

\usepackage[margin=1in]{geometry}
\usepackage{setspace}

\begin{document}
\begin{raggedleft}
	John Markiewicz \\
	3 May 2015 \\
	American Lit Accel. with Bucher \\
	Crucible Essay Draft \\
\end{raggedleft}
\begin{center}
A Story of Suspense
\end{center}

\doublespacing

	The drama has been used to inspire feelings in its audiences since its 
	invention.  A skilled playwright can play on the tendency of people to feel
	a certain way to change the way one looks at a play, and at the ideas it
	attempts to convey.  The best cases of this use a feeling in the audience to
	strengthen the piece in both plot and impact.  Suspense plays a central role
	in \underline{The Crucible} in its role in the plot, its effects on the
	characters, and the way Miller uses it to set the mood of the play.

	The plot of \underline{The Crucible} is driven in significant part by the 
	force of suspense.  Toward the beginning of the play, the atmosphere of
	suspense has the townspeople anxious.  Putnam says, with vicious
	certainty ``I'd not call it sick, the Devil's touch is heavier than sick.
	{''} (13).  Without this sense of anxiety---this worried suspense---the 
	people would never have worried about witchcraft in the first place.  Later,
	suspense drives many characters to bring accusations of witchcraft against 
	others.  Marry Warren is driven into a state of near hysteria, saying about 
	Sarah Good, ``Mr. Proctor, in open court she near to choked us all to death 
	\ldots She tried to	kill me many times, Goody Proctor!'' (57).  The suspense 
	in the novel builds further, leading to Proctor saying to Mary, ``You're 
	coming to court with me, Mary.  You will tell it in the court,'' to which Mary
	responds ``She'll kill me for sayin' that! \ldots Abby'll charge lechery on 
	you, Mr. Proctor''(80).	Throughout the novel the suspense drives the plot of 
	\underline{The Crucible}.

	Suspense shapes the actions and motivation of the characters throughout the 
	play.  Suspense over what will happen with the Proctors moves Abigail to 
	make sure ``A needle were found stuck in her belly---''(76).  This event was 
	used by Abigail to cast possible guilt on Goody Proctor.  The suspense
	over the possibility of witchcraft against Abigail and the warrant for Goody
	Proctor's arrest leads Cheever to say (on the matter of a poppet) ``Tis hard 
	proof!  I find here a poppet Goody Proctor keeps.  I have found it, sir\ldots
	I never wanted to see such proof of hell, and I bid you obstruct me not, for 
	I---''(75).  Later in the scene, Proctor can no longer handle the suspense and
	snaps, saying as he rips the warrant, ``Out with you \ldots Is the accuser 
	always holy now?  Were [Parris and Abigail] born this morning as clean as
	God's fingers? \ldots vengeance is walking Salem''(77).  Through these events
	and the rest of the play, suspense plays a critical role in the actions of
	the characters.

	Miller uses suspense not just as a device to develop the play in its plot and
	its characters, but also as a manner to set the mood of the play.  In the 
	beginning of the play, the stage direction reads ``Parris is praying now, and,
	though we cannot hear his words, a sense of confusion hangs about him\ldots 
	[Tituba] enters as one who can no longer bear to be barred from the sight of
	her beloved, but is also very frightened because\ldots as always, trouble in
	this house eventually lands on her back''(8).  Later in the play, the mood in
	the scene at the proctor's house is summed up by the direction: ``Enter
	Ezekiel Cheever.  A stunned silence''(72).  The whole scene is tense, stunned,
	and hostile as soon as Cheever enters the room.  Later in the trial the sense
	of suspense is driven even by the placement of characters, as ``As the curtain
	rises, the room is empty\ldots Through the partitioning wall at the right we 
	hear a prosecutor's voice, Judge Hathorne's\ldots then a woman's voice''(83).
	Through the play, suspense is key to the feeling the audience has and the
	tone the story sets.

	Suspense is used as an important device in \underline{The Crucible}, and suits
	the subject matter very well.  In a society where the opinion of the majority 
	has a strong sway on who is killed, the feeling of nervous suspense about an
	event such as a witch trial would be natural.  By using the natural
	feelings in a story, a writer can effectively convey a message and 
	entertain an audience.
\end{document}
