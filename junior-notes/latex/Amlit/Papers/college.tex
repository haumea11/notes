\documentclass[12pt]{article}
\usepackage[margin=1in]{geometry}
\usepackage{setspace}

\begin{document}
\begin{raggedleft}
	John Markiewicz\\
	23 March 2015\\ 
	Am. Lit. Acc with Bucher\\
	College Essay\\
\end{raggedleft}
\begin{center}
	Why I Wear a Suit
\end{center}
\doublespacing

	People usually have idiosyncracies, details about themselves
	that distinguish them from others.  A lot of these don't do
	anything to others' opinions, or actually influence them 
	negatively.  Some aspects of individuals, however, make
	a person different from the accepted norm in a good way.  I
	started wearing formal clothing as a challenge to the idea
	that it is uncool to do so, and it has benefitted me in 
	several ways.

	Generally speaking, we as a culture associate the suit
	with businessmen (and businesswomen.)  This gives it a whole
	set of connotations; people have an image spring to mind
	every time the word ``businessman'' is said.
	On one hand, business is associated with bureaucracy, 
	boredom, dissatisfaction, and of course work.  In a more
	positive respect, though, business is associated with success, 
	money, drive, and power in its more positive connotations.
	Additionally, suits are often worn as part of a uniform,
	which also has its own set of connotations: uniforms are 
	boring, and lack individuality, but allow greater unity
	of image and can imply a sense of dependability or regularity.
	My goal in wearing a suit was to take advantage of the
	positive aspects of the businesslike uniform look without 
	evoking the negative aspects, in an attempt to build a
	personal image.

	Originally, I had challenged the idea that suits are bland
	as a response to an unpleasant situation.  I had just made the
	move to high school, and was getting used to the ins and
	outs of my dress code.  I had become pretty bored with the
	routine of the days.  One Friday, since there was no rule
	against it, and since I was so bored, I wore a suit to school.
	The look caught on,	and so every Friday thereafter I wore a
	suit to keep the tradition going.  As people got used to it,
	they took a liking to it, and by the time summer rolled 
	around people were disappointed that I didn't wear suits
	on Fridays.

	People responded surprisingly well to my initial move
	toward formal clothes.  It was partially as a result of
	the suits that I got my first nickname, ``Crazy John.''
	Regardless of perspective on whether this nickname is
	a good thing, it was given with positive intent.  As a 
	result of how people responded, and how comfortable I
	have become with suits, I would absolutely challenge the
	idea that suits are boring and uncomfortable again.

	Suits are one way I differ from an accepted cultural 
	norm.  However, regardless of how closely they follow
	the widely accepted fashion, people always differ from
	their culture in some ways.  If it weren't for the
	differences of individuals from the norm, the world would
	end up a horribly boring place to be.  
\end{document}
