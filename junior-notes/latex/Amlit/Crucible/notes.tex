\documentclass[10pt]{article}

\usepackage{fullpage}

\begin{document}
\section{Background Information}
\subsection{American Drama}
\begin{itemize}
	\item Eugene O'neill was the first ``important'' figure in American Drama.  
	\item Introduced realistic drama with the removal of the viewer by the 
		fourth wall.
	\item Arthur Miller and Tennessee Williams represent Realism and Realism 
		combined 	with an attempt at something more imaginative.
	\item Miller wrote Death of a Salesman, All My Sons, and The Crucible.
\end{itemize}

\subsection{The Crucible}
\begin{itemize}
	\item Produced in 1953, about puritans and the salem witch trials.  
		This was in parallel to McCarthy's Red Hunt.  God, I hate McCarthy.
	\item There was a belief in Salem in witches, seen as the Devil's helpers,
		and	confession led to freedom from the devil's bonds.  Confusion in 
		identifying	witches came from shape-shifting and possession.
\end{itemize}

\section{Act One}
\begin{itemize}
	\item Reverend Samuel Parris is a main character, play opens in his room.
		He kneels by his daughter's bed, she is lying inert.
	\item Act opens with a digression, and a discussion of puritans.  One
		point he makes about these people tend to mind others' business, and
		that this feeds the witch trials.
	\item The puritans believe that they light the world.  This has helped
		and hurt America in the time since.
	\item Miller says that the Salem Witch Trial developed from a paradox
		in whose grip we still live.  The issue is one of the theocracy, but
		the balance turning toward individual freedom.
	\item We return to a mid-forties Parris, who is a bad man.
	\item Tituba is a slave, brought from Barbados by Parris from when he
		was a businessman there.  Believed to be able to conjure spirits.
	\item Abigail is strikingly beautiful, an orphan living with Parris
		who is her uncle.  She is very dramatic.  She tends to seem like
		she's lying here, because of the way she's acting.
	\item Someone was running naked in the forest.  Abigail may have drank 
		blood.  They danced.
	\item Betty is lying unresponsive in bed, tries to fly when she wakes
		up.  She actually passed out, probably.
	\item Mercy is a merciless.  She's the Putnam's servant, is sly.
	\item Mary is subservient, naive, and lonely.  She's a bit of a foil
		to Mercy.
	\item Mrs. Putnam tends to be full of canards.
	\item Abigail says she was discharged from Goody Proctor's service
		because she hates Abigail and has it out for her.  People want 
		slaves, not servants, so she hasn't gotten a job lately.
	\item It is assumed that the actual reason is that Abigail has
		had an affair with Goody Proctor's husband John Proctor.
	\item Ann Putnam has had many babies die.  She seems to be the gossip
		of the town, and simply makes matters worse.
	\item Thomas Putnam is a well-to-do, hard-handed land-owner and is
		50.  A commentary follows; he is a man with many grievances, at 
		least one of which appears to be justified.  His brother-in-law
		got turned down for a position he should have gotten.  Putnam is
		the eldest son of the richest man in the village.  His vindictive 
		nature was demonstrated long before the witchcraft happened at all.
	\item Putnam feels his own name and the honor of his family have been 
		besmirched by the village.  He feels he was unfairly treated in his
		father's will.  He accuses a lot of people.
	\item Putnam dislikes Parris, and Parris is not entirely aware of this.
	\item Putnam gives Parris advice to strike out against the devil,
		and this may just be a ploy to mess Parris up.
	\item Betty awakes, and tries to fly out the window.  Abigail drank
		blood, and Betty makes note of this.
	\item The girls danced, Tituba tried to conjure the spirits of 
		some dead sisters or something.  Abigail threatens the hell out of
		the other girls in case they do tell.
	\item ``In Proctor's presence, a fool felt his foolishness instantly.''
	\item Proctor is respected, and feared, but he feels like a fraud.
	\item Proctor has a specific attitude toward Abigail, and she's 
		pretty into him.  He still thinks about her, but states he will
		never touch her again.
	\item Rebecca Nurse and Giles Corey enter.
	\item Nurse is really calm, chill.  Her husband is well-respected,
		and she is as well.  She's been through a lot, and is viewed as
		wise.  She's very gentle, by nature.
	\item Putnams and Nurses have a land dispute.
	\item Putnam accuses Rebecca's spirit of tempting her to iniquity,
		and this is more true than Putnam knows.
	\item Nurse has 11 children, 26 grandchildren, and have seen all
		of them go through their foolishness.  She has a lot of experience
		with children and says you have to let them be kids.
	\item Parris has sent for Reverend Hale, who apparently has some
		expertise in the field of witches and devils.  Rebecca says he
		should just be sent back as soon as he gets here, saying that the
		Reverend will just foster further disagreement.  Mrs. Putnam seems
		to resent and envy Rebecca Nurse who has had so many living children
		and grandchildren.
	\item There are arguments about land, deeds, Parris's preaching,
		firewood, factions in the church.  This is a town full of conflict
		and contention.  Reverend Hale enters.
	\item The ideals of the world haven't changed since the days of strong
		religion even though there is a superficial change.
	\item Hale says not to look to superstition, and yet this is a witch
		trial play.
	\item Putnams tend to be witchcraft believers, Abigail, the other girls,
		etc.  Parris doesn't want witchcraft, and Proctor, Nurse, and maybe
		a couple others just don't believe it.
	\item Giles is wacky, comical, gets blamed for a lot, isn't very good
		at praying.  Was a crank, and a nuisance, but is deeply innocent and
		brave.
	\item Hale questions the girls, especially Tituba and Abigail.  
	\item Girls try to push the blame off themselves, for the most part
	\item More small talk is made, and a sense of separation rises.  
		Elizabeth doesn't want conflict but she must cause it anyway.
\end{itemize}

\section{Act Two}
\begin{itemize}
	\item Early in the act Mary Warren and the Proctors are talking about
		other girls.  Sarah Good and Sarah Osborne, in particular.  Warren 
		seems shaken up	over the whole situation.  She likes to be treated 
		like she means something, to be the centre of attention.
	\item Sarah Good ``Never knew any commandments'' and this shows the 
		witch hunt is toward anyone who isn't flawless in the eye of the
		Puritan.
	\item Mary is defiant, and then goes to sleep while Elizabeth fears.
		John says he'll find Ezekiel Cheever, Elizabeth feels they need more
		than Cheever.
	\item Elizabeth goes into a monologue about Abigail, who is actually
		pretty damn clever.  Abigail wants to take the place of Elizabeth
		with Proctor.
	\item Elizabeth gets no commentary from Miller.
	\item Elizabeth has fortitude, faith, tries to make the best of the
		situation.  Parris asks Abigail if her name is white in town, 
		and she says Goody Proctor says things about her, but there is no
		actual support for this yet.  Judge Hale shows up, and says that
		Elizabeth has been mentioned in the court.
	\item Hale begins to question them, and they get onto the subject
		of the Church, the candlesticks, and Parris.  Proctor could be
		described as an ``Independent Thinker''  as he doesn't think
		going to Church makes sense since Parris isn't preaching what he
		agrees with; Proctor thinks one must hold firm with what they
		believe in.
	\item Nurse has been under suspicion, and Hale says it is a pretty
		crazy time.
	\item Hale believes that since Parris is ordained, the light of God
		must be in him.  The commandments thing happens again.
	\item Hale tries to smile, and is prodding them.
	\item Elizabeth wants John to tell Hale about Abigail, and the
		conversation turns to her.  Giles Corey and Francis Nurse appear
		after that.  Both their wives have been taken as suspects.
	\item Hale says he knows Rebecca is a good person, and that the 
		court is likely to let her go.
	\item ``Until an hour before the Devil fell, God found him beautiful
		in Heaven.''
	\item Giles explains how his wife is charged by a man named Walcott
		accused his wife (Martha Corey) 
	\item Cheever and someone enter.
	\item Cheever believes you must do as you're told, which is quite the
		dangerous attitude to have.
	\item Cheever asks Elizabeth if she keeps poppets in her house, which
		is related to the poppet and Abigail.  Elizabeth was framed in this
		way.
	\item Things get heated, Proctor rips up the order.  Hale states that
		``Vengeance is walking in Salem'' and speaks the truth about how the
		place is being run by vengeance and children.
	\item Allusion is made to Pontius Pilate, and that Hale did the same
		as Pilate and washed his hands of something he could have stopped.
	\item Hale looks away with guilt and uncertainty as Elizabeth is taken
		off.  Corey calls to Hale, who then says he will testify in her
		favor.  
	\item Proctor says Hale is a coward despite being ordained in God's 
		tears.
	\item Hale says God wouldn't be provoked grandly by such a petty cause.
		Unwittingly references Proctor's affair.  
	\item Mary Warren won't charge Abigail.
	\item Proctor is willing to sink the ship he's living on just to kill
		the captain.
	\item Abigail plans to ``Save'' Proctor.
\end{itemize}

\section{Act Three}
\begin{itemize}
	\item Miller uses irony quite a bit.  Bucher's laptop presents the 
		Powerpoint he wanted to use here.
	\item Opens in the courtroom.  Characters are not seen, they are
		merely heard at the opening.
	\item The lack of people visible at the start of the scene implies a
		hiddenness and adds a tension.
	\item Martha Corey is on trial, and her husband Giles shows up to
		interrupt the court.  The charges have changed against her. 
	\item Giles says against Putnam that he is rigging the trial to rob
		the land from the convicted.
	\item Hathorne is a bitter, remorseless Salem judge.  
	\item Danforth didn't mean to accuse anyone of anything, but did 
		anyway.
	\item Hale says Giles claims hard evidence in defense of his wife.
		Danforth interrupts him and says the evidence must be submitted by
		a proper affidavit.
	\item Herrick offends Danforth by saying they are deceiving him.
		Danforth responds with 
	\item Notes here have been missed.  Not cool.
	\item Proctor confesses, and Danforth asks questions.  Abigail is a
		harlot or something.
	\item People are to turn their backs so they can't tell Abigail what to
		say.  Elizabeth is forced to lie.
	\item At the end of the act, Abigail and the other girls start to act 
		as though they're seeing spirits.  Mary tries to get them to stop.
	\item By the end of the act, Giles Corey is arrested for contempt of
		court and Proctor is arrested for witchcraft.
	\item Hale has changed and wants to be done with the procedures and
		the situation.
\end{itemize}
\section{Act Four}
\begin{itemize}
	\item Tituba and Sarah Good in jail, Herrick and Danforth come in.  Hale has 
		come back and he goes and prays among those who are going to hang.  Parris
		told Hale to do this.  Danforth is displeased that he's in the jail.
	\item Talk ends up turning to Parris, and that he has a mad look in his eyes
		sometimes.  He's unsteady.
	\item Cheever mentions all the cows wandering.
	\item Hale has apparently returned to ``Bring Rebecca Nurse to God.''
	\item Danforth says that it is indeed a providence, asking if the prisoners
		are softened.
	\item Parris tells Danforth that Abigail is missing, and that his strongbox
		has been broken into and all the money has been taken.
	\item Parris essentially says that since the people set to be executed are
		well-liked by the town.
	\item Parris hears a dagger hit the ground when opening his door.  The dagger
		was in his door.  Says ``You cannot hang this sort, there is danger for
		me.''
	\item Hale enters, and says they will not budge.  Danforth says he cannot
		pardon them when people have hung for the cause.
	\item Hale sort of freaks out, and says there is blood on his head (obviously
		a metaphorical blood.)
	\item Hale makes an allusion and then says to Elizabeth that she should cleave
		to no faith when faith brings blood.  Life is God's most precious gift and
		no principle should lead to ending it.  
	\item God damns a liar less than he who throws his life away for pride.  
	\item Elizabeth says this would be the devil's argument.
	\item Elizabeth reveals some things to Proctor, and vice versa.
	\item Giles Corey was pressed.  Old-school.
\end{itemize}

\end{document}
