%%=============================================================================
%%
%%       Filename:  mathHW.tex
%%
%%    Description:  The math homework, starting as of the 7th March 2015
%%
%%        Version:  1.0
%%        Created:  03/07/2015
%%       Revision:  none
%%
%%         Author:  John Markiewicz
%%   Organization:  
%%      Copyright:  
%%
%%          Notes:  
%%                
%%=============================================================================
\documentclass[11pt]{article}
\usepackage[]{amsmath}
\usepackage{fullpage}
\DeclareMathOperator{\arcsec}{arcsec}
\DeclareMathOperator{\arccot}{arccot}
\DeclareMathOperator{\arccsc}{arccsc}
\begin{document}
\section{Section 5}
\subsection{377/5-12, 17-27 odds (Section 5.6, day 1)}
\begin{enumerate}
	\item[5. ] Find the arcsin of $\frac{1}{2}$ without using a calculator.
		$$\arcsin \frac{1}{2} = \frac{\pi}{6}$$
	\item[6. ]  $\arcsin 0 = 0$
	\item[7. ]  $\arccos \frac{1}{2} = \frac{\pi}{3}$
	\item[8. ]  $\arccos 0 = \frac{\pi}{2}$
	\item[9. ]  $\arctan \frac{\sqrt{3}}{3} = \frac{\pi}{6}$
	\item[10. ] $\arccot (-\sqrt{3}) = -\frac{\pi}{6}$
	\item[11. ] $\arccsc \left(-\sqrt{2}\right) = \arcsin \frac{1}{(-\sqrt{2})} 
		= -\frac{\pi}{4}$
	\item[12. ] $\arccos \left(-\frac{\sqrt{3}}{2}\right) = \frac{5\pi}{6}$
	\item[17. ] Evaluate without using a calculator:
		\begin{enumerate}
			\item \[\sin \left(\arctan \frac{3}{4}\right) = \frac{3}{5}\]
			\item \[\sec \left(\arcsin \frac{4}{5}\right) = \frac{5}{3}\]
		\end{enumerate}
	\item[21. ] Write in ``Algebraic'' form.
		\[\cos \left(\arcsin 2x\right) = 2x\]
	\item[23. ] Write in ``Algebraic'' form. 
		\[\sin (\arcsec x) = \frac{\sqrt{x^2 - 1}}{|x|}\]
	\item[25. ] Write in ``Algebraic'' form.
		\[\tan\left(\arcsec\frac{x}{3}\right) = \frac{\sqrt{x^2-9}}{3}\]
	\item[27. ] Write in ``Algebraic'' form.
		\[\csc\left(\arctan \frac{x}{\sqrt{2}}\right) = \frac{\sqrt{x^2+2}}{x}\]
	\end{enumerate}
\subsection{378/41-65 e.o.o. (Section 5.6, day 2)}
\begin{enumerate}
	\item[41. ] Find the derivative.
		\begin{align}
			f(x) &= 2 \arcsin (x-1) \\
			\frac{dy}{dx} &= \frac{2}{\sqrt{1-(x-1)^2}} \\
			%% ^ This line is by the derivative of arcsin as given in 5.6
			&= \frac{2}{\sqrt{1-(x^2-2x+1)}} \\
			&= \frac{2}{\sqrt{2x-x}}
		\end{align}
	\item[45. ] Find the derivative.
		\begin{align}
			\setcounter{equation}{0}
			f(x) &= \arctan \frac{x}{a} \\ %We're assuming, a is a constant.
			%% Again, we use the derivative formulae.  We've seen these
			%% done by others, but this is an implementation where it's much
			%% faster to do it using the formula.
			\frac{dy}{dx} &= \frac{\frac{1}{a}}{(\frac{x}{a})^2+1} \\
			&= \frac{1}{a\left(\frac{x}{a}\right)^2+a} \\
			&= \frac{1}{\frac{x^2}{a}+a} \\
			&= \frac{a}{x^2+a^2}
			%% Should get mention tomorrow.  Not sure if it really makes sense 
			%% for this to be a^2.
		\end{align}
	\item[49. ] Find the derivative.
		\begin{align}
			\setcounter{equation}{0}
			f(x) &= \sin (\arccos t) \\
			\frac{dy}{dx} &= \left[\cos (\arccos t)\right] 
			\left[\frac{-1}{\sqrt{1-t^2}}\right] \\
			&= \frac{-\cos(\arccos t)}{\sqrt{1-t^2}} \\
			&= \frac{-t}{\sqrt{1-t^2}}
		\end{align}
	\item[53. ] Find the derivative.
		\begin{align}
			\setcounter{equation}{0}
			y &= \frac{1}{2}\left[\frac{1}{2}\ln\frac{x+1}{x-1}+\arctan x\right] \\
			&= \frac{1}{4}\ln\frac{x+1}{x-1}+\frac{1}{2}\arctan x \\ 
			y &= \frac{1}{4}\ln (x+1) - \frac{1}{4}\ln (x-1) + \frac{1}{2}\arctan x\\
			\frac{dy}{dx} &= 1\left(\frac{1}{4x+4}\right) - 
			1\left(\frac{1}{4x-4}\right) + \frac{1}{2x^2+2} \\
			&= \frac{1}{4x+4} - \frac{1}{4x-4} + \frac{1}{2x^2+2}
		\end{align}
		\\ Notes: I had issue with these problems.  Here's the solution as
		given in class:
	\item[53. ] Again.
		\begin{align}
			\setcounter{equation}{0}
			y &= \frac{1}{2}\left[\frac{1}{2}\ln\frac{x+1}{x-1}+\arctan x\right] \\
			&= \frac{1}{4}\ln(x+1) - \frac{1}{4} \ln (x-1) + \frac{1}{2}\arctan x \\
			\frac{dy}{dx} &= \frac{1}{4(x+1)} -\frac{1}{4(x-1)}+\frac{1}{2(x^2+1)} \\
			&= \frac{-x^2-1+x^2-1}{2(x^2+1)(x^2-1)} \\
			&= \frac{-1}{x^4-1}
		\end{align}
	\item[57. ]
		\begin{align}
			\setcounter{equation}{0}
			y &= \frac{1}{2}\left[\frac{1}{2}\ln\frac{x+1}{x-1}+\arctan x\right] \\
			\frac{dy}{dx} &= \frac{2}{\sqrt{1-\left(\frac{x}{4}\right)}} 
			- \frac{1}{2}\left[\sqrt{16-x^2} + \frac{-2x^2}{2\sqrt{16-x^2}}\right] \\
			&= \frac{2}{\sqrt{\frac{16-x^2}{16}}} - \frac{1}{2}\left[\sqrt{16-x^2
			+ \frac{-2x^2}{2\sqrt{16-x^2}}}\ \right] \\
			&= \frac{8}{\sqrt{16-x^2}} - \frac{\sqrt{16-x^2}}{2} + \frac{x^2}
			{2\sqrt{16-x^2}} \\
			&= \frac{16 - (16-x^2)+x^2}{2\sqrt{16-x^2}} \\
			&= \frac{x^2}{\sqrt{16-x^2}}
		\end{align}
	\end{enumerate}
	\subsection{385/1-41 e.o.o. ; 47; 63}
	\begin{enumerate}
		\item[1. ] Find the Integral.
			\begin{align}
				\setcounter{equation}{0}
				\int \frac{5}{\sqrt{9-x}} dx
			\end{align}
		\item[5. ] Find the Integral.
			\begin{align}
				\setcounter{equation}{0}
				\int \frac{1}{x\sqrt{4x^2-1}}dx
			\end{align}
		\item[9. ] Find the Integral.
			\begin{align}
				\setcounter{equation}{0}
				\int \frac{1}{\sqrt{1-(x+1)^2}} dx
			\end{align}
		\item[13. ] Find the Integral.
			\begin{align}
				\setcounter{equation}{0}
				\int \frac{e^{2x}}{4+e^{4x}} dx
			\end{align}
		\item[17. ] Find the Integral
			\begin{align}
				\setcounter{equation}{0}
				\int \frac{x-3}{x^2+1}dx
			\end{align}
		\item[21. ] Evaluate the integral.
			\begin{align}
				\int_{0}^{1/6} 
	\end{enumerate}
\end{document}
